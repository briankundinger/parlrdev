% Options for packages loaded elsewhere
\PassOptionsToPackage{unicode}{hyperref}
\PassOptionsToPackage{hyphens}{url}
\PassOptionsToPackage{dvipsnames,svgnames*,x11names*}{xcolor}
%
\documentclass[
  12pt,
]{article}
\usepackage[]{mathpazo}
\usepackage{setspace}
\usepackage{amssymb,amsmath}
\usepackage{ifxetex,ifluatex}
\ifnum 0\ifxetex 1\fi\ifluatex 1\fi=0 % if pdftex
  \usepackage[T1]{fontenc}
  \usepackage[utf8]{inputenc}
  \usepackage{textcomp} % provide euro and other symbols
\else % if luatex or xetex
  \usepackage{unicode-math}
  \defaultfontfeatures{Scale=MatchLowercase}
  \defaultfontfeatures[\rmfamily]{Ligatures=TeX,Scale=1}
\fi
% Use upquote if available, for straight quotes in verbatim environments
\IfFileExists{upquote.sty}{\usepackage{upquote}}{}
\IfFileExists{microtype.sty}{% use microtype if available
  \usepackage[]{microtype}
  \UseMicrotypeSet[protrusion]{basicmath} % disable protrusion for tt fonts
}{}
\makeatletter
\@ifundefined{KOMAClassName}{% if non-KOMA class
  \IfFileExists{parskip.sty}{%
    \usepackage{parskip}
  }{% else
    \setlength{\parindent}{0pt}
    \setlength{\parskip}{6pt plus 2pt minus 1pt}}
}{% if KOMA class
  \KOMAoptions{parskip=half}}
\makeatother
\usepackage{xcolor}
\IfFileExists{xurl.sty}{\usepackage{xurl}}{} % add URL line breaks if available
\IfFileExists{bookmark.sty}{\usepackage{bookmark}}{\usepackage{hyperref}}
\hypersetup{
  colorlinks=true,
  linkcolor=blue,
  filecolor=Maroon,
  citecolor=Blue,
  urlcolor=Blue,
  pdfcreator={LaTeX via pandoc}}
\urlstyle{same} % disable monospaced font for URLs
\usepackage[margin=1in]{geometry}
\usepackage{graphicx}
\makeatletter
\def\maxwidth{\ifdim\Gin@nat@width>\linewidth\linewidth\else\Gin@nat@width\fi}
\def\maxheight{\ifdim\Gin@nat@height>\textheight\textheight\else\Gin@nat@height\fi}
\makeatother
% Scale images if necessary, so that they will not overflow the page
% margins by default, and it is still possible to overwrite the defaults
% using explicit options in \includegraphics[width, height, ...]{}
\setkeys{Gin}{width=\maxwidth,height=\maxheight,keepaspectratio}
% Set default figure placement to htbp
\makeatletter
\def\fps@figure{htbp}
\makeatother
\setlength{\emergencystretch}{3em} % prevent overfull lines
\providecommand{\tightlist}{%
  \setlength{\itemsep}{0pt}\setlength{\parskip}{0pt}}
\setcounter{secnumdepth}{-\maxdimen} % remove section numbering
\usepackage{hyperref}
\usepackage[]{natbib}
\bibliographystyle{plainnat}

\author{}
\date{\vspace{-2.5em}}

\begin{document}

\setstretch{1}
\begin{flushright} 
    \end{flushright}
    \begin{center} \textbf{Base Model Specification}
    
    Brian Kundinger

    \end{center}

\hypertarget{notation-and-assumptions}{%
\subsection{Notation and Assumptions}\label{notation-and-assumptions}}

Our notation and assumptions closely follow that of Sadinle (2017).
Denote two files as \(\mathbf{X}_A\) and \(\mathbf{X}_B\), with \(n_A\)
and \(n_B\) records respectively, and with records indexed as
\(i \in \{1, \ldots, n_A\}\) in \(\mathbf{X}_A\) and
\(j \in \{1, \ldots, n_B\}\) in \(\mathbf{X}_B\). Without loss of
generality, label the files such that \(n_A \geq n_B\). We also assume
there are no duplicates within files, only across. For each record pair
under consideration, we generate a comparison vector
\(\boldsymbol{\gamma}_{ij} = \{\gamma_{ij}^1, \ldots, \gamma_{ij}^F\}\),
where \(F\) is the number of fields used in the linkage and each takes
\(\gamma_{ij}^f\) takes on a value \(l \in \{1, \ldots, L_f\}\)
indicating level agreement between the two records on a specified field.

Following the Fellegi Sunter framework, we define
\(m^{fl}:= P(\gamma_{ij}^f = l |Z_j = i)\) to be the probability of
observing agreement level \(l\) in field \(f\) for records \(i\) and
\(j\) given that the records are a match, and similarly define
\(u^{fl}:= P(\gamma_{ij}^f = l |Z_j \neq i)\), for non-matches. We
denote \(\lambda\) to be the (marginal) probability that some record
\(j \in \mathbf{X}_B\) has a match in \(\mathbf{X}_A\).

Lastly, we adopt from Sadinle 2017 the matching labelling
\(\mathbf{Z} = (Z_1, \ldots, Z_{n_A})\), defined as \[Z_j=\begin{cases} 
    i,  & \text{if records } i\in \mathbf{X}_A \text{ and } j\in \mathbf{X}_B \text{ refer to the same entity}; \\
    n_A + 1,  & \text{if record } j\in \mathbf{X}_B \text{ does not have a match in file} \mathbf{X}_B; \\
\end{cases}\]

Wherever possible, we reserve superscripts for denoting field, level,
and linkage cluster, and reserve subscripts for record indices. For
example, \(\mathbf{m}^f = (m^{f1}, \ldots, m^{fL_f})\) is the
probability distribution governing field \(f\) for matching records, and
\(\mathbf{m}_{ij}= \prod_{f=1}^{F}\prod_{l=1}^{L_f} \left(m^{fl}\right)^{\mathbf{1}_{\gamma_{ij}^f = l}} = P(\boldsymbol{\gamma}_{ij}|Z_j = i)\)
is product of the relevant of the appropriate \(\mathbf{m}\) parameters
for record pair \((i,j)\). We hope that these conventions avoid
overloaded notation in the likelihood and subsequent derivations.

\hypertarget{model-specification}{%
\section{Model Specification}\label{model-specification}}

Additionally, our model differs from that of Sadinle 2017 through its
explicit dependence on a beta random variable \(\lambda\) that models
the rate of matching across records. Sadinle marginalizes over such a
random variable in his derivations of the ``beta prior for bipartite
matching,'' but here we provide derivations without marginalizing in
order to specify differing rates of matching for different linkage
clusters.

\hypertarget{prior-distributions-and-likelihood}{%
\subsection{Prior Distributions and
Likelihood}\label{prior-distributions-and-likelihood}}

For fields \(f \in \{1, \ldots, F\}\) and levels
\(l\in \{1, \ldots, L_f\}\) we adopt the following likelihood and prior
distributions. Note that likelihood is related to that of Sadinle
(2017), but includes dependence on the \(\lambda\) random variable. It
also makes it explict that units of observation under this model are the
\(n_B\) records in \(\mathbf{X}_B\), a crucial distinction when compared
to the likelihood over the \(n_A \times n_B\) record pairs in the
original Fellegi Sunter model.

\[P(\Gamma|\mathbf{Z}, \mathbf{m}, \mathbf{u}, \lambda) =\prod_{j=1}^{n_B}  \prod_{i=1}^{n_A}\mathbf{m}_{ij}^{\mathbf{1}_{z_j = i}}\mathbf{u}_{ij}^{\mathbf{1}_{z_j \neq i}}\]

\[\mathbf{m^{f}} \sim \text{Dirichlet}(\alpha^{f1}, \ldots, \alpha^{fL_f})\]
\[\mathbf{u^{f}} \sim \text{Dirichlet}(\beta^{f1}, \ldots, \beta^{fL_f})\]
\[Z_j | \lambda =
\begin{cases} 
    \frac{1}{n_A}\lambda  & z_j \leq n_A; \\
     1-\lambda &  z_j  = n_A + 1 \\
\end{cases}\]

\[\lambda \sim \text{Beta}(\alpha_{\lambda}, \beta_{\lambda}) \] The
prior for \(Z_j\) has equal probability of matching to all records
\(i\in \mathbf{X}_A\), and non-matching probability governed by
\(\lambda\). Therefore a \(\lambda \sim \text{Beta}(1, 1)\) corresponds
to a prior belief that nonmatches and matches are equally likely, and a
\(\lambda \sim \text{Beta}(1, \frac{1}{n_A})\) prior corresponds to a
uniform prior on the labelling of \(\mathbf{Z}\).

\hypertarget{posterior-sampling}{%
\subsection{Posterior Sampling}\label{posterior-sampling}}

We work with the following factorization of the joint distribution:

\[p(\Gamma, \mathbf{Z}, \mathbf{m}, \mathbf{u}, \lambda) = p(\Gamma|\mathbf{Z}, \mathbf{m}, \mathbf{u}) p(\mathbf{Z} | \lambda) p(\mathbf{m}, \mathbf{u}) p(\lambda)\]

This factorization leads to following Gibbs Sampler:

\underline{Sample $\mathbf{m}^{(s+1)}$ $\mathbf{u}^{(s+1)}|\Gamma, \mathbf{Z}^{(s)}$:}
The \(\mathbf{m}\) and \(\mathbf{u}\) parameters are updated through
standard multinomial-dirichlet mechanics. Thus we have

\[\mathbf{m^{f}}|\mathbf{Z}, \Gamma \sim \text{Dirichlet}(\alpha^{f1}(\mathbf{Z}), \ldots, \alpha^{fL_f}(\mathbf{Z}))\]
\[\mathbf{u^{f}}|\mathbf{Z}, \Gamma \sim \text{Dirichlet}(\beta^{f1}(\mathbf{Z}), \ldots, \beta^{fL_f}(\mathbf{Z}))\]
where
\(\alpha_{fl}(\mathbf{Z})= \sum_{i,j} I_{obs}(\gamma_{ij}^f)\mathbf{1}_{\gamma_{ij}^f = l} \mathbf{1}_{z_j = i}\)
and
\(\beta_{fl}(\mathbf{Z})= \sum_{i,j} I_{obs}(\gamma_{ij}^f)\mathbf{1}_{\gamma_{ij}^f = l} \mathbf{1}_{z_j \neq i}\).

\textbackslash underline\{Sample \(\lambda^{(s+1)}|\mathbf{Z}^{(s)}\) As
a function of \(\lambda\), the linkage structure parameter
\(\mathbf{Z}\) is sequence of successes (when \(z_j < n_A + 1\)) and
failures (when \(z_j = n_A + 1\)), and therefore
\(p(\mathbf{Z}|\lambda) = \mathcal{L}(\lambda|\mathbf{Z})\) is
determined only by the number of dupliates
\(D = \sum_{i=1}^{n_B}\mathbf{1}_{z_j < n_A + 1}\) encoded by
\(\mathbf{Z}\). Thus we have

\[p(\lambda | \mathbf{Z}) \propto p(\mathbf{Z}|\lambda)p(\lambda)\]
\[\propto \lambda^D (1-\lambda)^{n_B - D} \lambda^{\alpha_{\lambda} -1} (1-\lambda)^{\beta_{\lambda} -1}\]
\[ \propto \lambda^{D + \alpha_{\lambda} - 1} (1-\lambda)^{n_B - D + \beta_{\lambda} -1}\]
\[\implies \lambda^{(s+1)}|\mathbf{Z}^{(s+1)} \sim \text{Beta}(D + \alpha_{\lambda}, n_B - D + \beta_{\lambda})\]

\underline{Sample $\mathbf{Z}^{(s+1)}|\Gamma, \mathbf{m}^{(s+1)}, \mathbf{u}^{(s+1)}, \lambda^{(s+1)}$}
Because we sample \(Z_j\) independently of all other \(Z_{j'}\), we use
only the full conditional for an individual \(Z_j\). Let \(\Gamma_{.j}\)
denote the set of \(n_A\) comparison vectors with \(j \in B\), and note
that as a function of \(Z_j\), the likelihood
\(p(\Gamma_{.j}|Z_j, \mathbf{m}, \mathbf{u})\) is a discrete
distribution with probalities proportional to
\[p(\Gamma_{.j}|Z_j = z_j, \mathbf{m}, \mathbf{u}) \propto \prod_{i=1}^{n_A}\mathbf{m}_{ij}^{\mathbf{1}_{z_j = i}}\mathbf{u}_{ij}^{\mathbf{1}_{z_j \neq i}}\]
\[\propto \prod_{i=1}^{n_A}\left(\frac{\mathbf{m}_{ij}}{\mathbf{u}_{ij}}\right)^{\mathbf{1}_{z_j = i}} \;\;\;\;\;\;\;\; \text{Divide through by} \prod_{i = 1}^{n_A}\mathbf{u}_{ij}\]
\[=
\begin{cases} 
    w_{ij}  & z_j \leq n_A; \\
    1 &  z_j  = n_A + 1 \\
\end{cases}\]

where
\(w_{ij} = \frac{\mathbf{m}_{ij}}{\mathbf{u}_{ij}} = \frac{P(\boldsymbol{\gamma_{ij}}|Z_j = i)}{P(\boldsymbol{\gamma_{ij}} |Z_j \neq i)}\).
The interested reader should note that these are precisely the
likelihood ratios used in the Fellegi-Sunter model to classify matches
and non-matches, and we therefore refer to \(w_{ij}\) as the
\emph{Fellegi Sunter weights}.

With the likelihood in this form, we can derive the full conditional
\[p(Z_j|\Gamma_{.j}, \mathbf{m} ,\mathbf{u}, \lambda) \propto p(\Gamma_{.j}| Z_j, \mathbf{m} ,\mathbf{u}) P(Z_j|\lambda)\]

\[\propto \left(\sum_{i=1}^{n_A}w_{ij}\mathbf{1}_{z_j = i} + \mathbf{1}_{z_j = n_A + 1}\right)\left(\lambda\sum_{i=1}^{n_A}\frac{1}{n_A}\mathbf{1}_{z_j = i} + (1-\lambda)\mathbf{1}_{z_j = n_A + 1}\right)\]
\[= \frac{\lambda}{n_A}\sum_{i=1}^{n_A}w_{ij}\mathbf{1}_{z_j = i} + (1-\lambda)\mathbf{1}_{z_j = n_A + 1} \]
\[ \implies Z_j^{(s+1)} | \mathbf{m}, \mathbf{u}, \Gamma, \lambda \propto
\begin{cases} 
    \frac{\lambda}{n_A}w_{ij}   & z_j \leq n_A; \\
     1-\lambda &  z_j  = n_A + 1 \\
\end{cases}\]

Here, one should note that if we choose to integrate over the posterior
of \(\lambda\) and rearrange terms, we produce the following sampling
mechanism:

\[Z_j^{(s+1)} | \mathbf{m}, \mathbf{u}, \mathbf{Z^{(s)}} \propto
\begin{cases} 
    w_{ij}  & z_j \leq n_A; \\
     n_A \frac{n_B - D + \beta_{\lambda}}{D + \alpha_{\lambda}} &  z_j  = n_A + 1 \\
\end{cases}\]

\end{document}
