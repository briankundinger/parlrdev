% Options for packages loaded elsewhere
\PassOptionsToPackage{unicode}{hyperref}
\PassOptionsToPackage{hyphens}{url}
\PassOptionsToPackage{dvipsnames,svgnames*,x11names*}{xcolor}
%
\documentclass[
  12pt,
]{article}
\usepackage[]{mathpazo}
\usepackage{setspace}
\usepackage{amssymb,amsmath}
\usepackage{ifxetex,ifluatex}
\ifnum 0\ifxetex 1\fi\ifluatex 1\fi=0 % if pdftex
  \usepackage[T1]{fontenc}
  \usepackage[utf8]{inputenc}
  \usepackage{textcomp} % provide euro and other symbols
\else % if luatex or xetex
  \usepackage{unicode-math}
  \defaultfontfeatures{Scale=MatchLowercase}
  \defaultfontfeatures[\rmfamily]{Ligatures=TeX,Scale=1}
\fi
% Use upquote if available, for straight quotes in verbatim environments
\IfFileExists{upquote.sty}{\usepackage{upquote}}{}
\IfFileExists{microtype.sty}{% use microtype if available
  \usepackage[]{microtype}
  \UseMicrotypeSet[protrusion]{basicmath} % disable protrusion for tt fonts
}{}
\makeatletter
\@ifundefined{KOMAClassName}{% if non-KOMA class
  \IfFileExists{parskip.sty}{%
    \usepackage{parskip}
  }{% else
    \setlength{\parindent}{0pt}
    \setlength{\parskip}{6pt plus 2pt minus 1pt}}
}{% if KOMA class
  \KOMAoptions{parskip=half}}
\makeatother
\usepackage{xcolor}
\IfFileExists{xurl.sty}{\usepackage{xurl}}{} % add URL line breaks if available
\IfFileExists{bookmark.sty}{\usepackage{bookmark}}{\usepackage{hyperref}}
\hypersetup{
  colorlinks=true,
  linkcolor=blue,
  filecolor=Maroon,
  citecolor=Blue,
  urlcolor=Blue,
  pdfcreator={LaTeX via pandoc}}
\urlstyle{same} % disable monospaced font for URLs
\usepackage[margin=1in]{geometry}
\usepackage{graphicx}
\makeatletter
\def\maxwidth{\ifdim\Gin@nat@width>\linewidth\linewidth\else\Gin@nat@width\fi}
\def\maxheight{\ifdim\Gin@nat@height>\textheight\textheight\else\Gin@nat@height\fi}
\makeatother
% Scale images if necessary, so that they will not overflow the page
% margins by default, and it is still possible to overwrite the defaults
% using explicit options in \includegraphics[width, height, ...]{}
\setkeys{Gin}{width=\maxwidth,height=\maxheight,keepaspectratio}
% Set default figure placement to htbp
\makeatletter
\def\fps@figure{htbp}
\makeatother
\setlength{\emergencystretch}{3em} % prevent overfull lines
\providecommand{\tightlist}{%
  \setlength{\itemsep}{0pt}\setlength{\parskip}{0pt}}
\setcounter{secnumdepth}{-\maxdimen} % remove section numbering
\usepackage{hyperref}
\usepackage[]{natbib}
\bibliographystyle{plainnat}

\author{}
\date{\vspace{-2.5em}}

\begin{document}

\setstretch{1}
\begin{flushright} 
    \end{flushright}
    \begin{center} \textbf{Parlr: Parallelized Record Linkage in R}
    
    Brian Kundinger

    \end{center}

\hypertarget{introduction}{%
\subsection{Introduction}\label{introduction}}

Most methods for conducting record linkage, the task of identifying
duplicate records across datasets, are derived from the foundational
article by Fellegi and Sunter in 1969. Within this family of methods,
two methods have notable strengths: \texttt{fastlink}, proposed by
Enamorado et al (2019), continues a is able to conduct linkage on
incredibly large datasets

\hypertarget{notation-and-assumptions}{%
\subsection{Notation and Assumptions}\label{notation-and-assumptions}}

Our notation and assumptions closely follow that of Sadinle (2017).
Denote two files as \(A\) and \(B\), with \(n_A\) and \(n_B\) records
respectively, and with records indexed as \(i \in \{1, \ldots, n_A\}\)
in \(A\) and \(j \in \{1, \ldots, n_B\}\) in \(B\). Without loss of
generality, label the files such that \(n_A \geq n_B\). We also assume
there are no duplicates within files, only across. For each record pair
under consideration, we generate a comparison vector
\(\boldsymbol{\gamma}_{ij} = \{\gamma_{ij}^1, \ldots, \gamma_{ij}^F\}\),
where \(F\) is the number of fields used in the linkage and each
\(\gamma_{ij}^f\) takes on a value \(l \in \{1, \ldots, L_f\}\)
indicating level agreement between the two records on a specified field.

To indicate matching status, we adopt the \emph{linkage structure
parameter} \(\mathbf{Z} = (Z_1, \ldots, Z_{n_B})\) from Sadinle 2017,
defined as \[Z_j=\begin{cases} 
    i,  & \text{if records } i\in A \text{ and } j\in B \text{ refer to the same entity}; \\
    n_A + 1,  & \text{if record } j\in B \text{ does not have a match in file } A; \\
\end{cases}\] This provides more memory efficient storage for the
linkage information than a \(n_A \times n_B\) sparse matrix of
indicators.

Following the Fellegi Sunter framework, we define
\(m^{fl}:= P(\gamma_{ij}^f = l |Z_j = i)\) to be the probability of
observing agreement level \(l\) in field \(f\) for records \(i\) and
\(j\) given that the records are a match, and similarly define
\(u^{fl}:= P(\gamma_{ij}^f = l |Z_j \neq i)\), for non-matches. We also
adopt Fellegi and Sunter's conditionally independent fields assumption
that the level of agreement on one field is independent of the level of
agreement on another. Though this assumption is often not reasonable
(for example, first name and gender are two clearly dependent fields),
but it is common within the record linkage literature and generally
leads to models that perform well in practice; see discussion for
further remarks. Lastly, we define \(\lambda\) to be the (marginal)
probability that some record \(j \in B\) has a match in \(A\).

Wherever possible, we reserve superscripts for denoting field and level,
while reserving subscripts for record indices. For example,
\(\mathbf{m}^f = (m^{f1}, \ldots, m^{fL_f})\) is the probability
distribution governing field \(f\) for matching records, and
\(\mathbf{m}_{ij}= \prod_{f=1}^{F}\prod_{l=1}^{L_f} \left(m^{fl}\right)^{\mathbf{1}_{\gamma_{ij}^f = l}} = P(\boldsymbol{\gamma}_{ij}|Z_j = i)\)
is product of the relevant of the appropriate \(\mathbf{m}\) parameters
for record pair \((i,j)\). We hope that these conventions avoid
overloaded notation in the likelihood and subsequent derivations.

\hypertarget{model-specification}{%
\section{Model Specification}\label{model-specification}}

(This will be revised to talk more explicitly about sampling \(Z_j\)
independently)

Our model differs from that of Sadinle 2017 through its explicit
dependence on a beta random variable \(\lambda\) that models the rate of
matching across records. Sadinle marginalizes over such a random
variable in his derivations of the ``beta prior for bipartite
matching,'' but here we provide derivations without marginalizing in
order to specify differing rates of matching for different linkage
clusters.

\hypertarget{prior-distributions-and-likelihood}{%
\subsection{Prior Distributions and
Likelihood}\label{prior-distributions-and-likelihood}}

For fields \(f \in \{1, \ldots, F\}\) and levels
\(l\in \{1, \ldots, L_f\}\) we adopt the following likelihood and prior
distributions.

\[P(\Gamma|\mathbf{Z}, \mathbf{m}, \mathbf{u}, \lambda) =\prod_{j=1}^{n_B}  \prod_{i=1}^{n_A}\mathbf{m}_{ij}^{\mathbf{1}_{z_j = i}}\mathbf{u}_{ij}^{\mathbf{1}_{z_j \neq i}}\]

\[\mathbf{m^{f}} \sim \text{Dirichlet}(\alpha^{f1}, \ldots, \alpha^{fL_f})\]
\[\mathbf{u^{f}} \sim \text{Dirichlet}(\beta^{f1}, \ldots, \beta^{fL_f})\]
\[Z_j | \lambda =
\begin{cases} 
    \frac{1}{n_A}\lambda  & z_j \leq n_A; \\
     1-\lambda &  z_j  = n_A + 1 \\
\end{cases}\]

\[\lambda \sim \text{Beta}(\alpha_{\lambda}, \beta_{\lambda}) \] The
prior for \(Z_j\) has equal probability of matching to all records
\(i\in A\), and non-matching probability governed by \(\lambda\).
Therefore a \(\lambda \sim \text{Beta}(1, 1)\) corresponds to a prior
belief that nonmatches and matches are equally likely, and a
\(\lambda \sim \text{Beta}(1, \frac{1}{n_A})\) prior corresponds to a
uniform prior on the labeling of \(\mathbf{Z}\).

\hypertarget{posterior-sampling}{%
\subsection{Posterior Sampling}\label{posterior-sampling}}

We work with the following factorization of the joint distribution:

\[p(\Gamma, \mathbf{Z}, \mathbf{m}, \mathbf{u}, \lambda) = p(\Gamma|\mathbf{Z}, \mathbf{m}, \mathbf{u}) p(\mathbf{Z} | \lambda) p(\mathbf{m}, \mathbf{u}) p(\lambda)\]

This factorization leads to following Gibbs Sampler:

\underline{Sample $\mathbf{m}^{(s+1)}$ $\mathbf{u}^{(s+1)}|\Gamma, \mathbf{Z}^{(s)}$:}
The \(\mathbf{m}\) and \(\mathbf{u}\) parameters are updated through
standard multinomial-dirichlet mechanics. Thus we have

\[\mathbf{m}^f|\mathbf{Z}, \Gamma \sim \text{Dirichlet}(\alpha^{f1}(\mathbf{Z}), \ldots, \alpha^{fL_f}(\mathbf{Z}))\]
\[\mathbf{u}^f|\mathbf{Z}, \Gamma \sim \text{Dirichlet}(\beta^{f1}(\mathbf{Z}), \ldots, \beta^{fL_f}(\mathbf{Z}))\]
where
\(\alpha_{fl}(\mathbf{Z})= \sum_{i,j} \mathbf{1}_{obs(\gamma_{ij}^f)}\mathbf{1}_{\gamma_{ij}^f = l} \mathbf{1}_{z_j = i}\)
and
\(\beta_{fl}(\mathbf{Z})= \mathbf{1}_{obs(\gamma_{ij}^f)}\mathbf{1}_{\gamma_{ij}^f = l} \mathbf{1}_{z_j \neq i}\).

\underline{Sample $\lambda^{(s+1)}|\mathbf{Z}^{(s)}$:} As a function of
\(\lambda\), the linkage structure parameter \(\mathbf{Z}\) is sequence
of successes (when \(z_j < n_A + 1\)) and failures (when
\(z_j = n_A + 1\)), and therefore
\(p(\mathbf{Z}|\lambda) = \mathcal{L}(\lambda|\mathbf{Z})\) is
determined only by the number of duplicates
\(D = \sum_{i=1}^{n_B}\mathbf{1}_{z_j < n_A + 1}\) encoded by
\(\mathbf{Z}\). Thus we have

\[p(\lambda | \mathbf{Z}) \propto p(\mathbf{Z}|\lambda)p(\lambda)\]
\[\propto \lambda^D (1-\lambda)^{n_B - D} \lambda^{\alpha_{\lambda} -1} (1-\lambda)^{\beta_{\lambda} -1}\]
\[ \propto \lambda^{D + \alpha_{\lambda} - 1} (1-\lambda)^{n_B - D + \beta_{\lambda} -1}\]
\[\implies \lambda^{(s+1)}|\mathbf{Z}^{(s+1)} \sim \text{Beta}(D + \alpha_{\lambda}, n_B - D + \beta_{\lambda})\]

\underline{Sample $\mathbf{Z}^{(s+1)}|\Gamma, \mathbf{m}^{(s+1)}, \mathbf{u}^{(s+1)}, \lambda^{(s+1)}$:}
Because we sample \(Z_j\) independently of all other \(Z_{j'}\), we use
only the full conditional for an individual \(Z_j\). Let \(\Gamma_{.j}\)
denote the set of \(n_A\) comparison vectors with \(j \in B\), and note
that as a function of \(Z_j\), the likelihood
\(p(\Gamma_{.j}|Z_j, \mathbf{m}, \mathbf{u}) = \mathcal{L}(Z_j|\Gamma_{.j}, \mathbf{m}, \mathbf{u})\)
is a discrete distribution with probabilities proportional to

\begin{align*}
p(\Gamma_{.j}|Z_j = z_j, \mathbf{m}, \mathbf{u}) &\propto \prod_{i=1}^{n_A}\mathbf{m}_{ij}^{\mathbf{1}_{z_j = i}}\mathbf{u}_{ij}^{\mathbf{1}_{z_j \neq i}}\\
&\propto \prod_{i=1}^{n_A}\left(\frac{\mathbf{m}_{ij}}{\mathbf{u}_{ij}}\right)^{\mathbf{1}_{z_j = i}} && \text{By dividing through by} \prod_{i = 1}^{n_A}\mathbf{u}_{ij}\\
&=
\begin{cases} 
    w_{ij}  & z_j \leq n_A; \\
    1 &  z_j  = n_A + 1 \\
\end{cases}\\
\end{align*}

where
\(w_{ij} = \frac{\mathbf{m}_{ij}}{\mathbf{u}_{ij}} = \frac{P(\boldsymbol{\gamma_{ij}}|Z_j = i)}{P(\boldsymbol{\gamma_{ij}} |Z_j \neq i)}\).
The interested reader should note that these are precisely the
likelihood ratios used in the Fellegi-Sunter model to classify matches
and non-matches, and we therefore refer to \(w_{ij}\) as the
\emph{Fellegi Sunter weights}.

With the likelihood in this form, we can derive the full conditional
\[p(Z_j|\Gamma_{.j}, \mathbf{m} ,\mathbf{u}, \lambda) \propto p(\Gamma_{.j}| Z_j, \mathbf{m} ,\mathbf{u}) P(Z_j|\lambda)\]

\[\propto \left(\sum_{i=1}^{n_A}w_{ij}\mathbf{1}_{z_j = i} + \mathbf{1}_{z_j = n_A + 1}\right)\left(\lambda\sum_{i=1}^{n_A}\frac{1}{n_A}\mathbf{1}_{z_j = i} + (1-\lambda)\mathbf{1}_{z_j = n_A + 1}\right)\]
\[= \frac{\lambda}{n_A}\sum_{i=1}^{n_A}w_{ij}\mathbf{1}_{z_j = i} + (1-\lambda)\mathbf{1}_{z_j = n_A + 1} \]
\[ \implies Z_j^{(s+1)} | \mathbf{m}, \mathbf{u}, \Gamma, \lambda \propto
\begin{cases} 
    \frac{\lambda}{n_A}w_{ij}   & z_j \leq n_A; \\
     1-\lambda &  z_j  = n_A + 1 \\
\end{cases}\]

In order to make fair comparisons against the Sadinle 2017 model, we
integrate over the posterior of \(\lambda\) and rearrange terms to
produce the final full conditional:

\[p\left(Z_j^{(s+1)}  = i| \mathbf{m}, \mathbf{u}, \mathbf{Z^{(s)}}\right) \propto
\begin{cases} 
    w_{ij}  & i \leq n_A; \\
     n_A \frac{n_B - D + \beta_{\lambda}}{D + \alpha_{\lambda}} & i  = n_A + 1 \\
\end{cases}\]

\hypertarget{bayes-estimate}{%
\subsection{Bayes Estimate}\label{bayes-estimate}}

Our Gibbs sampler provides posterior samples of \(\mathbf{Z}\) which we
use to make our final decisions about the linkage structure. In the case
where we false matches and missed matches contribute the same loss, we
declare \((i,j)\) to be a match whenever \(P(Z_j = i) > \frac{1}{2}\)
according to these posterior samples. We can create more elaborate
decisions by attributing different loss values to different kinds of
error, and allowing pairings with middling posterior probabilities to be
left unlabeled by the algorithm so that the modeler can address those
manually. For further discussion, see Sadinle 2017.

\hypertarget{efficient-computation}{%
\section{Efficient Computation}\label{efficient-computation}}

Broadly speaking, we increase our computational efficiency by
recognizing that record pairs contribute to posterior calculations only
through the agreement pattern of the \(\gamma_{ij}\) vector. Let \(H\)
be the set of unique agreement patterns in the data, let \(P\) denote
the total number of unique agreement patterns. Note that \(P\) is
bounded above by \(\prod_{f=1}^F L_f\), and that this bound does not
scale with \(n_A\) or \(n_B\). We index these agreement patterns by
\(p \in \{1, \ldots, P\}\), and say \((i,j) \in h_p\) when the \((i,j)\)
pair exhibits the \(p^{th}\) agreement pattern. Wherever possible, we
conduct calculations over these \(P\) agreement patterns rather than the
\(n_A \times n_B\) record pairs.

\hypertarget{data-representation-and-hashing}{%
\subsection{Data Representation and
Hashing}\label{data-representation-and-hashing}}

In the classic Fellegi Sunter framework, \(\Gamma\) is a
\(n_A n_B \times F\) matrix, with each row providing the comparison
vector for a different \((i,j)\) pair. We however do not store these
comparison vectors themselves, but instead only the a hashed value
corresponding to the agreement pattern of the \((i, j)\) pair. Enamorado
et al (2019) provided the hashing function
\[\sum_{f=1}^F \mathbf{1}_{\gamma_{(i,j)}^f >0}2^{\gamma_{(i,j)}^f + \mathbf{1}_{f>1} \times \sum_{e=1}^{f-1}(L_e-1)}\]
to map each agreement pattern to a unique value, but packages like
\texttt{dplyr} in \texttt{R} are capable of this as well.

We store this information in a nested list \(\tilde{\Gamma}\) where the
\(p^{th}\) component of the \(j^{th}\) list contains a vector of records
in \(A\) that share agreement pattern \(h_p\) with record \(j \in B\).
For each \(p\), we also calculate
\(|h_p| = \sum_{i=1}^{n_A}\sum_{j=1}^{n_B} \mathbf{1}_{(i,j) \in h_p}\)
total instances of agreement pattern \(h_p\) throughout the data, and
also for each \(j\), we calculate
\(|h_p|_j = \sum_{i=1}^{n_A} \mathbf{1}_{(i,j) \in h_p}\) the instances
of agreement pattern \(p\) among the comparison vectors between record
\(j \in B\) and each of the \(n_A\) records in \(A\).

For large data, we can partition the two datasets \(A\) and \(B\) into
smaller blocks \(\{A_m\}\) and \$\{B\_m\} for more manageable
computations. On a single machine, we can read-in data sequentially,
conduct hashing, collect results, and delete the original data from
memory before continuing with the next chunk of data. With multiple
cores or multiple machines, this can be done in parallel. Storing this
hashed information still becomes burdensome for large data, but this
hashing method greatly expands the capabilities of the Fellegi-Sunter
framework.

Lastly, the classic Fellegi Sunter method represents the \(\gamma_{ij}\)
comparison vector as vector of length \(F\), with each component
\(\gamma_{ij}^f\) taking on values in \(\{0, \ldots, L_f - 1\}\). To
ease computations, we instead represent the comparison as a
concatenation of \(F\) many binary indicator vectors of lengths \(L_f\).
For example, if \(L_1 = L_2 = 2\) and \(L_3 = 3\), then
\(\gamma_{ij} = (1, 0, 2)\) under the classical framework becomes
\(\gamma_{ij} = (0, 1, 1, 0, 0, 0, 1)\) under our framework. This is a
bijective transformation that does not change the meaning of the data,
but this representation will ease calculations and posterior updates.

\hypertarget{gibbs-sampling}{%
\subsection{Gibbs Sampling}\label{gibbs-sampling}}

After receiving matching statuses from \(\mathbf{Z}\), the Sadinle
method calculates \(\alpha_{fl}(\mathbf{Z})\) and
\(\beta_{fl}(\mathbf{Z})\) for each field and level. This constitutes
\(2 \times \sum L_f\) many summations over \(n_A \times n_B\)
quantities, and becomes computationally burdensome with large data. In
contrast, we recognize that each unique agreement pattern contributes to
the posterior \(\alpha(\mathbf{Z})\) and \(\beta(\mathbf{Z})\) vectors
in the same way. In fact, if we denote
\(|h_p^m| = \sum_{j=1}^{n_B} \mathbf{1}_{(Z_j, j) \in h_p}\) to be the
number of matching record pairs with agreement pattern \(h_p\), then the
contribution of pairs of pattern \(h_p\) to the \(\alpha(\mathbf{Z})\)
vector is simply \(\mathbf{Z}(h_p) \times h_p\). Thus our posterior
update for \(\alpha\) is simply
\(\alpha(\mathbf{Z}) = \alpha_0 + \sum_{p=1}^P |h_p^m| \times h_p\).
Then, we can easily calculate \(|h_p^u|\), the number of nonmatching
record pairs of agreement pattern \(p\), by subtracting the number of
matching pairs from the total present in the data; that is
\(|h_p^u| = |h_p| - |h_p^m|\). From this, we can update our \(\beta\)
parameter through
\(\beta(\mathbf{Z}) = \beta_0 + \sum_{p=1}^P |h_p^u| \times h_p\). Note
that these constitute \(P\) many summations over \(n_B\) quantities, and
thus avoids the \(n_A \times n_B\) summation from the original method.

Sadinle uses a prior for \(\mathbf{Z}\) that induces the a full
conditional for \(Z_j\) that strictly enforces one-to-one matching.
Particular, this sampler removed previously matches records from the set
of candidate records when sampling \(Z_j\), creating a dependency that
makes the sampler \emph{inherently serial}.By weakening the one-to-one
requirement, our full conditional for \(Z\) does not depend on the rest
of the \(\mathbf{Z}_{-j}\) vector, and thus can be computed in parallel.
More importantly, since only the agreement pattern of \(Z_j\) is used
for calculations within the Gibbs sampler, and not the particular record
label, we can conduct this sampling only at the level of the unique
agreement patterns. This boosts computation time far greater than
parallelization.

To do this we calculate the Fellegi Sunter weight \(w_{h_p}\) for each
unique pattern, sample the agreement pattern between \(j\) and its
potential match, and then sample the record label uniformly among viable
records. More concretely, define \(H(Z_j)\) to be the agreement pattern
between \(j\) and its potential match, and say \(H(Z_j) = h_{P+1}\) when
\(Z_j = n_A + 1\). Then,

\[H\left(Z_j^{(s+1)}\right) | \mathbf{m}, \mathbf{u}, \mathbf{Z^{(s)}} \propto
\begin{cases} 
    w_{h_p}\times |h_p|_j  & h(z_j) \in H; \\
     n_A \frac{n_B - D + \beta_{\lambda}}{D + \alpha_{\lambda}} &  z_j  = n_A + 1 \\
\end{cases}\]

We complete the entire Gibbs procedure at the level of the \(P\)
agreement patterns. After, we can back-fill the records corresponding to
the agreement patterns by sampling uniformly at random among candidate
records stored in \(\tilde{\Gamma}\).

\hypertarget{enforcing-bipartite-matching}{%
\subsection{Enforcing Bipartite
Matching}\label{enforcing-bipartite-matching}}

Add this

\hypertarget{simulation-studies}{%
\section{Simulation Studies}\label{simulation-studies}}

\begin{itemize}
\item
  Replicating the Sadinle Study
\item
  \texttt{parlr} and \texttt{BRL} speed comparison
\end{itemize}

\hypertarget{data-analysis}{%
\section{Data Analysis}\label{data-analysis}}

\begin{itemize}
\tightlist
\item
  Necessary? El Salvador data? Or perhaps needs to be a bigger dataset?
\end{itemize}

\hypertarget{discussion}{%
\section{Discussion}\label{discussion}}

We note that this often not a reasonable assumption; for example if we
know that two records differ on gender, this does not give any
information on whether their year of births are the same, but it does
suggest that its more likely that their first names are different.

\end{document}
