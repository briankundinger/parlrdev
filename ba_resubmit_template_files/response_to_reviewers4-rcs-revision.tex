\documentclass[letterpaper, parskip]{scrartcl}

\usepackage{amsmath,amssymb,bm}
\usepackage{url}
\usepackage[hidelinks]{hyperref}
\usepackage[margin=1.4in]{geometry}
\usepackage{natbib}
\usepackage{xcolor}
\usepackage{tcolorbox}            % provides shaded box for reviewer comments
\usepackage{enumitem}             % use more compact lists
%\usepackage[parfill]{parskip}     % use skips between paragraphs and no indents
\usepackage{bibentry}             % provides \nobibliography command
\usepackage{xspace}

% Make itemize and enumerate environments more compact
\setlist[itemize]{noitemsep, topsep=0pt, parsep=0pt}
\setlist[enumerate]{noitemsep, topsep=0pt, parsep=0pt}

% -------- Define point raised command --------- %

\definecolor{darkpurple}{HTML}{500050}
\tcbuselibrary{breakable}       % allow tcolorbox to break across pages
\tcbuselibrary{skins}
\tcbset{%
	enhanced,%
	frame hidden,%
	sharp corners,%
	breakable,%
	parbox=false,%
	colback=lightgray!30,%
	enlarge left by=6pt,%
	width=\linewidth-6pt,%
	colback=gray!0,
	left=4pt,%
	right=0pt,%
	top=0pt,%
	bottom=0pt,%
	borderline west={2pt}{0pt}{lightgray},%
	%coltext=darkpurple,%
	before skip=14pt plus 2pt,%
	after skip=14pt plus 2pt}

%\newcommand{\pointRaised}[1]{%
%	\begin{tcolorbox}
%		\itshape #1
%	\end{tcolorbox}
%}
%
%\newcounter{responsectr}[section]     % counter for response resets at each section
%\newcommand{\reply}[2]{%
%	\refstepcounter{responsectr}%
%	\textbf{#1.\theresponsectr:} #2
%}

\newcommand{\pointRaised}[2]{%
	\textbf{#1.\theresponsectr:} #2
}

\newcounter{responsectr}[section]     % counter for response resets at each section
\newcommand{\reply}[1]{%
	\refstepcounter{responsectr}%
		\begin{tcolorbox}
			\itshape #1
		\end{tcolorbox}
}


\newcommand{\todo}{\textcolor{red}{[TODO]}\xspace}


\newcommand{\beka}[1]{{\color{teal}}}


\begin{document}

	% ****************** TITLE ****************************************

	\title{Response to Reviewer Comments}
	\maketitle
	
We thank the Editor and Associate Editor for their assessment and constructive feedback.

\section{Editor Comments}

\pointRaised{E}{I received a detailed and careful report from the original associate editor who thinks the paper is now easier to read and understand. I still find it intricate to read, but the AE may have a point when saying that the notation can be a difficult challenge in this area of research.}
\reply{}


\pointRaised{E}{Anyhow, I have decided to follow the AE’s recommendation and would like the authors to carefully consider the comments in the AE report and make the suggested revisions. I find the issue raised by the AE regarding the formulas in Section 4.2 particularly concerning for two reasons. First, the article cannot be accepted if it contains technical errors and inaccuracies. Second, the formulas in question were used to code up the samplers used in the simulations and data analysis examples. If the formulas are incorrect (do not make sense, in fact, as the AE points out) how can the computational results be trusted? This is a grave concern that must be satisfactorily and convincingly addressed before the manuscript is accepted for publication.}
\reply{}

\pointRaised{E}{Please prepare a revision addressing all the AE’s comments, explain how the errors in the formulas affected the computations, and revise the computations as needed.}
\reply{}
%Thank you for the thoughtful feedback. We have made several broad changes to address the Editor's and Associate Editor's overall critiques, and have addressed each specific comment from the reviewer.

\section{Associate Editor Comments}

\textcolor{red}{Brian: change E to AE. Make sure to address the reviewer as AE (many typos below). Address the questions/comments of the Editor above.}

\pointRaised{E}{For Section 4.2, my question starts with the definition of $\Gamma_{Z_{j^{(s+1)}, j}}.$ Is this quantity always defined for possible values of $Z$? That is, if $Z_{j}^{(s+1)} = n_A + j$, what is the meaning of $\Gamma_{n_A+j, j}$?}

\reply{The reviewer is correct that $\Gamma_{n_A+j, j}$ is not defined under this notation. We had attempted to remove all instances in which this notation might be implied, but it seems we did not catch them all. We thank the editor for their thoughtful suggestions in this regard.
}

\pointRaised{E}{This notation first arises in the definition of $n_p(Z)$, which I think could easily be changed to something along the lines of $n_p(Z) = \sum_{j:Z_j \leq n_A} I(\gamma_{Z_j, j} = h_p)$.
}

\reply{Thank you for this suggestion. We have adopted this change.
}

\pointRaised{E}{My bigger concern comes for equation (18), where I am not entirely sure what probability is being computed, or how this sampling procedure combined with equation (19) would result in a “non-match” where $Z_j = n_A + j$. I believe that the idea is to partition the possible values of $Z_j:\left( \{r_{p_j}\}, n_A + j\right)$. Then, in equation (18), the left hand side is intended to be the probability that Z falls into each part of the partition. But, in the revision, I didn’t see where the last “non-match” part of the partition is defined, and so there is no real way to select the non-match from equation (19).
}

\reply{The editor's understanding of the sampler is correct. The editor also correctly identified an error in equation (19), such that there was no real way to select the non-match option. We thank the editor for the recommended change in notation, and have adopted the recommendation. Equations (18) and (19) now read:
	
	\begin{align*}
		p\left( Z_j^{(s+1)} \in r \mid \tilde{\gamma}, \bm{m}^{(s+1)}, \bm{u}^{(s+1)}, \pi^{(s+1)}\right) \propto
		\begin{cases} 
			\frac{\pi^{(s+1)}N_{p_j}}{n_A}  w_{p}^{(s+1)},  & r = r_{p_j}; \\
			1- \pi^{(s+1)} , &   r = n_A + j. \\
		\end{cases}
	\end{align*}
and

\begin{align*}
	p\left(Z_j^{(s+1)} = q \mid Z_j^{(s+1)} \in r, \bm{m}^{(s+1)}, \bm{u}^{(s+1)}, \pi^{(s+1)} \right) = \begin{cases} 
		\frac{1}{N_{p_j}}, & r = r_{p_j} \text{ and } q \in r; \\
		1, & r = n_A + j. \\
	\end{cases}
\end{align*} 
}

\pointRaised{E}{Also, it’s not clear to me as currently written that the authors are equating the patterns with that partition so that by definition a “true” match that does not appear in A cannot have a pattern that appears in $\mathcal{P}$, but did not happen to be observed (because the match wasn’t captured in $A$).
}

\reply{We did not quite understand this part of the comment. If record $B_j$ has no match in $A$, then $Z_j = n_A + j$. If there is no matching record, there is no comparison vector (or pattern) for such a non-existent record. 
	
Despite the confusion about this specific sentence, the proposed change in notation more explicitly shows how the sampler handles non-matches, which we think is the editors primary concern. 
}

\pointRaised{E}{Page 9, just below equation (17): Should the reference to equation (9) should be updated to equation (13)?}

\reply{The reference to equation (9) was meant as a comparison with equations (18) and (19), not equation (17). We have revised the paragraph to make this more clear. 
}

\textcolor{red}{Could you add in the new paragraph so it's more clear to the AE and for us to review please?}

\pointRaised{E}{Supplement A, just below equation (3): Should the summation be over index $j$?
}

\reply{Yes. We have made the correction. 
}

\pointRaised{E}{Supplement A, equation (12): I believe that this intermediate result is not described correctly. Because $c_j$ is a function of $\gamma_{ij}$, once the function is divided by $c_j$, the quantity is no longer proportional to the probability. Rather, it is proportional to a likelihood-like function. Because equation (13) is conditional on $\gamma$, this distinction is not important for the end result.
}

\reply{Since $\Gamma_{.j}$ is the data, the expression $p(\Gamma_{.j} | -)$ is intended to be interpreted as a likelihood. To avoid this confusion, and to be consistent with notation in equation (1) of the Supplement, and equations (15) and (17) in the main paper, we have revised equations (8) - (12) of the supplement to use $\mathcal{L}(Z_j|-)$ instead of $p(\Gamma_{.j}|-)$.
	
{
\color{red}{I added the changed notation in red. However, it relies on fairly ugly (and I think non-standard) uses of likelihood notation (for example, $\mathcal{L}(Z_j = q | -)$). In contrast, Jody's paper stayed with $p(\Gamma_{.j} | -)$ notation. Should we make the switch, or stand by the old notation? The reviewer acknowledged that it practically doesn't make a difference, so I'm perhaps inclined to stick with the version we submitted?
}
}
}

\textcolor{red}{I believe the AE is stating that things should be equality and no proportionality. If you look at what he/she wrote up this seems to be what he/she is stating to be the case. In essence, we cannot drop the constants. Please let me know if you agree regarding this Jerry as it's worded a bit strangely to me.}

	\clearpage
	
	\bigskip
	
	\bibliographystyle{jasa}
	\bibliography{biblio}

\end{document}
