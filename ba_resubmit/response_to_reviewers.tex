\documentclass[letterpaper, parskip]{scrartcl}

\usepackage{amsmath,amssymb,bm}
\usepackage{url}
\usepackage[hidelinks]{hyperref}
\usepackage[margin=1.4in]{geometry}
\usepackage{natbib}
\usepackage{xcolor}
\usepackage{tcolorbox}            % provides shaded box for reviewer comments
\usepackage{enumitem}             % use more compact lists
%\usepackage[parfill]{parskip}     % use skips between paragraphs and no indents
\usepackage{bibentry}             % provides \nobibliography command
\usepackage{xspace}

% Make itemize and enumerate environments more compact
\setlist[itemize]{noitemsep, topsep=0pt, parsep=0pt}
\setlist[enumerate]{noitemsep, topsep=0pt, parsep=0pt}

% -------- Define point raised command --------- %

\definecolor{darkpurple}{HTML}{500050}
\tcbuselibrary{breakable}       % allow tcolorbox to break across pages
\tcbuselibrary{skins}
\tcbset{%
	enhanced,%
	frame hidden,%
	sharp corners,%
	breakable,%
	parbox=false,%
	colback=lightgray!30,%
	enlarge left by=6pt,%
	width=\linewidth-6pt,%
	colback=gray!0,
	left=4pt,%
	right=0pt,%
	top=0pt,%
	bottom=0pt,%
	borderline west={2pt}{0pt}{lightgray},%
	%coltext=darkpurple,%
	before skip=14pt plus 2pt,%
	after skip=14pt plus 2pt}

%\newcommand{\pointRaised}[1]{%
%	\begin{tcolorbox}
%		\itshape #1
%	\end{tcolorbox}
%}
%
%\newcounter{responsectr}[section]     % counter for response resets at each section
%\newcommand{\reply}[2]{%
%	\refstepcounter{responsectr}%
%	\textbf{#1.\theresponsectr:} #2
%}

\newcommand{\pointRaised}[2]{%
	\textbf{#1.\theresponsectr:} #2
}

\newcounter{responsectr}[section]     % counter for response resets at each section
\newcommand{\reply}[1]{%
	\refstepcounter{responsectr}%
		\begin{tcolorbox}
			\itshape #1
		\end{tcolorbox}
}


\newcommand{\todo}{\textcolor{red}{[TODO]}\xspace}

\newcommand{\ncomment}[1]{{\color{teal} NM: #1}}


\begin{document}

	% ****************** TITLE ****************************************

	\title{Response to Reviewer Comments}

	\maketitle
	\textbf{We thank the Editor and reviewers for their encouraging assessment and constructive
	feedback.}


	\section*{Editor}


	\pointRaised{E}{%
	The paper has been reviewed by a referee and an associate editor and I have read the manuscript independently, before looking at the reports.  When I confront my notes and overall impression with the reviewers’ comments, I find myself in substantial agreement with the reviewers.

	As indicated by both reviewers, the work expands on previous work by Sadinle, introducing computational advances that are achieved by relaxing some requirements in the original formulation of the problem.  The close connections with the previous work and the extent of the novel contributions are not adequately explained.  This makes it hard for the reader to understand what is new and why the new contributions are valuable.  A careful editing of the presentation is needed to address this shortcoming.}

	\reply{%
\textbf{We thank the Editor, the AE, and the reviewer for their feedback, and pointing out our strengths of the paper and areas that we can improve the quality of our paper. We have revised our paper in the abstract and introduction as to make our contributions more clear in relation to prior work.  Furthermore, we have carefully edited the paper to address issues that were pointed out by the Editorial Board.} \todo
	}

	\newpage

	\pointRaised{E}{%
	Regarding the presentation, the AE points out a number of problems with notation, typos, and various inconsistencies.  Independently, I found many similar problems.  The intersection between what I found and what the AE found is non-empty and so is the symmetric difference.  Also, the AE found some issues that I did not find and vice versa.  So, I am almost certain that an additional review would uncover more problems.  Now, the results presented in the paper seem plausible as do the broad strokes of the derivations, but I must admit that I was not able to follow all the details. This, in large part, was because of the issues that I just mentioned, and I believe that most readers would find it difficult to follow the developments, as these issues are exceedingly distracting.
	}
	
	\reply{
\textbf{We appreciate these comments, and have done our utmost to address the ones listed by yourself and the AE. Furthermore, we have carefully gone through the paper ourselves, where we have corrected other issues with notation, typos, and various inconsistencies.} \todo
	}
	
	\pointRaised{E}{%
	There are problems with the notation, which in places is not defined, in others is used before being defined, and in others yet is used inconsistently. For example, $I_{obs()}$ is never defined, $n_{12(Z)}$ first appears on p. 5, but is not defined (in passing) until the bottom of p. 7, and it seems to become D on p. 8.
	}
	
	\reply{
\textbf{$I_{obs}()$ is now defined. We have defined $n_{12}(Z)$ at its first occurrence (and removed the error with $D$). We have checked for other notational inconsistencies.
	}
}
	
	\pointRaised{E}{%
	Throughout the manuscript, $n_1$ and $n_2$ are used interchangeably with $n_A$ and $n_B$, sometimes in the same section, as it happens, for example, in Section 3.1. 
	}
	
	\reply{
	\textbf{We have fixed the inconsistency, using $n_1$ and $n_2$ throughout the paper.}
	
	}
	
		\pointRaised{E}{%
	The loss function on p. 9 is out of Sadinle’s paper and it inherits the typo therein $(\theta_{11}, \text{if} \; Z_j, \;\hat{Z_j}, \ldots)$.
	}
	
	\reply{
	\textbf{The previous expression $Z_j, \hat{Z}_j \leq n_1$ was meant to be understood as $Z_j\leq n_1, \hat{Z}_j \leq n_1.$ We agree that the latter expression is more clear. This has been updated in our revision.}
	}
	
		\pointRaised{E}{%
	In the statement of Lemma 1 on p.13, B comes out of nowhere.  (Thinking I had missed something, I went back to the previous pages only to find out that B would be then defined in the proof.)
	}
	
	\reply{
	\textbf{We have updated the Lemma to include the definition of $B$ as well as other variables in the Lemma.}
	}
	
        	\pointRaised{E}{%
	Formatting of all displayed math must be checked and fixed, especially as far as punctuation is concerned, as commas seem to be missing in multiple places. For example, in Equation (1), there should be a comma after 0 and after 1, and this is by no means an isolated occurrence.
	}
	
	\reply{
		Punctuation for all math has been checked.
	}

\pointRaised{E}{%
	Confusion can also arise from lack of clarity in the exposition.  Take comment 4 from the AE, for example.  Whose marriage certificates and whose birth records are involved?  What records are in $X_1$ and what records are in $X_2$?  The reader should not be expected to go back to Newcombe’s paper to clarify the issue.
	}

	\reply{%
\textbf{We agree with this comment, and we have removed this example as it does not add to the paper.}
}

\pointRaised{E}{%
	Other comments that were raised are as follows: This is not an exhaustive list, but only a set of examples meant to point out what makes the paper hard to read and how it can be improved.  We summarize these below. 
}

\reply{	
	\begin{enumerate}
	\item p. 12, second paragraph:  insert “is” after Section 3.1.  \textbf{This has been fixed.}
	\item p. 15:  the first sentence of the last full paragraph is either missing a verb (“we observe”?) or it includes an extra “that.” \textbf{This has been fixed.}
	\item As I said, I do not believe that we, as reviewers, were able to uncover all the problems, and I strongly encourage the authors to do their part, as they should, to improve the presentation and eliminate all typos and inconsistencies. \textbf{This has been done by all authors.} \todo
	\item The AE finds the simulation studies to be incomplete and gives detailed suggestions on how those should be improved.  The AE also gives important suggestions about other aspects of the manuscript that must be carefully considered and addressed.  All of the AE’s comments are right on the mark. \textbf{We have provided detailed comments to the AE.}
	\end{enumerate}
	}

%	\reply{%
%	\begin{enumerate}
%		\item This change has been made
%		
%		\item \textcolor{red}{I could not find this error}
%		
%		\item We have read the paper for typos and inconsistencies, and thank the reviewers for bringing these to our attention. 
%		
%		\item We have responded to the AE's comments about the simulation studies
%	\end{enumerate}
%}

\pointRaised{E}{To summarize, the paper contains an interesting algorithmic contribution that can speed up calculations at the expense of relaxing some of the modeling conditions, without much adverse impact on the resulting inferences.  This aspect should be emphasized in a revised presentation.  The presentation should also make clearer the close connections to the relevant work by Sadinle, and all the issues that I mentioned above should be ironed out.
	
The authors should prepare a careful and substantive revision that remains within the editorial limit of 25 pages and that answers the various comments satisfactorily.  The authors should upload the revised manuscript together with a document detailing how they addressed the reviewers’ comments.}

\reply{%
	\textbf{Thank you for the opportunity to submit a revision and improve the quality of our paper. We have provided detailed responses to the reviewers' comments below, which prompted significant changes in our paper.} \todo
}

%\textcolor{blue}{Suggestions/Tips for Brian to Discuss: Let's work on the easy fixes first. Go fix all the notation, check the grammar, and fix the point by point responses of the Editor as these can be done hopefully quickly. Let's use this more organized format so we can more quickly talk about what comment the Editor is making. It also makes it easier for the Editor/AE to review, so they like this (just a pointer). Suggest putting lines in the paper so we can also know where we are and it's often helpful to put changes in color so we can check every detail before resubmission. These are just some tips that have worked in the past that I hope that will help. I typically like to handle the big changes last as they take a lot of time and more attention to detail. Antidote is a nice tool for checking grammar or asking someone who is really great at proofing to find these issues and mark it up for you. Olivier is pretty good at spotting such things or Ted, so they might be good choices. Finally, given that we have some sloppy stuff going on, I'd recommend having this really polished before sending it to HRDAG or Patrick Ball will think it's a mess. I personally don't like sending things that aren't polished out the door, and this my fault for not catching this prior to submission.}

	\clearpage
	\newpage

	\section*{Reviewer 1}

%\textcolor{blue}{Subtle comment to Brian: The Editor told us that this review is the AE, so we want to satisfy all these comments if possible since the Editor thinks the comments are "spot on"}


	\setcounter{responsectr}{0}

	\pointRaised{R1}{%
	This manuscript addresses the bipartite matching problem in data linkage and proposes a new fast computation version of “Beta Linkage” to allow such matching to take place for pairs of large databases without resorting to ad hoc blocking. It is built upon the Felegi-Sunter model of record linkages, where linkages are made within strata of agreement between records across the two databases, and relies heavily on advances made by Sadinle (2017). Overall, the paper presents a practical implementation of a modified Beta Linkage approach to	matching. It demonstrates that even though the new method suffers from the limitations of not seamlessly enforcing one-to-one matching, it is feasible for large datasets with modest numbers of features on which to match. Though the novelty of the presented methodology is primarily in the computational approach, I believe it to be a real advance for practical implementation of record linkage in modern applications.
	
	I have no major concerns about the content or accuracy of the presentation. However, substantial improvements are warranted to confirm my impression. In addition to the recommendation made by Reviewer 1 to include more clarity of the novel contributions of this manuscript, I believe that the explanation of the method should be written more clearly, the simulation comparisons should be repeated/expanded, and various features of the simplification should be fleshed out a bit more. Finally, more practical advice would be welcome. All of these points are more specifically described in the itemized list below
	}

	\reply{%
	\textbf{We thank the reviewer (AE) for noticing our computational novelties that expand upon the prior limitations of the prior work of Sadinle (2017). Our revised paper clarifies our novel contributions, improves upon the exposition of the writing, provides practical guidance for users, and expands the simulations. We expand on these points further below.} \todo
	}

	\pointRaised{R1}{%
		Overall, the notation is, somewhat by necessity, quite extensive. However, I’m not convinced that it’s entirely consistent. A glossary in the appendix may help the reader navigate the paper more easily.}
\reply{%
	We have provided a summary of notation in Appendix 7.1. 
	
	\textcolor{blue}{I would recommend a table for notation of the model and a table for notation of your hashing method. Why? It will be strange to reference it all together. I would try and put these in the main body if possible as it will really help the reader. You could also write a graphical model with notation of your model to help the reader. These are just some thoughts to think about. (I'm going to leave this up to you Brian if you want to abandon the able as it's quite annoying).}

	\textcolor{red}{Done}
}

\pointRaised{R1}{%
	Section 2, line 1: It may help to explicitly define X1 and X2 as being vectors of indices as implied by equations (1) and (2). 
}
\reply{
	We thank the reviewer for the suggestion, which has been incorporated into the paper in the following way: ``Consider two databases $X_1$ and $X_2$ containing records $\{x_{1i}\}_{i=1}^{n_1}$ and $\{x_{2j}\}_{j=1}^{n_2}$ respectively. Without loss of generality, denote files such that $n_1 \geq n_2$. We follow the convention established by Sadinle (2017) and say ``record i $\in X_1$" rather than the more compact $x_{1i}$ in order to avoid double subscripts."
	
%	\textcolor{red}{OPTION 2: We agree that the notation provided conflates the records themselves with the record indices. However, this notation was established in Sadinle 2017 and has been replicated in further work (Wortman 2019, Alishen-Guendel, one of Jerry's students), and we have decided to use it as well for consistency within the literature.}
%	\textcolor{blue}{We appreciate the suggestion, however, we believe it would be confusing to the existing prior work by Fellegi and Sunter (1969) that has been established as well as others that have extended the work, such as Sadinle (2014+), Wortman (2019), and Asheshin-Gundel (2020+). Given that it's widely used in this literature, we continue to utilize it.}
}

\pointRaised{R1}{%
Section 2.2, beta distribution: the $n_{12}(.)$ notation needs to be defined.
}
\reply{%
	\textbf{This has been defined.}
}

\pointRaised{R1}{%
Section 3, 4th full paragraph: My understanding of the example of matching birth records and
marriage certificates (presuming monogamy) may not quite be consistent with the algorithm as
presented, as I would think that one certificate (element in X1) would match multiple birth
records (elements in X2) which violates the size ordering. Can the algorithm be easily adjusted
for this reverse sizing?
}

\reply{
	\textbf{Thank you for the interesting comment/question. In this paper, our main goal and applications deal with bipartite record linkage (such that we can make fair comparison with prior work), which is consistent with the algorithm that we present in the paper. We have removed the confusing text mentioned by the reviewer. We are working on extensions beyond bipartite record linkage in other work, where the reverse sizing is not an issue; however, this is beyond the scope of this paper.}
}

%\textcolor{red}{I need some help addressing this in the paper. The answer is yes, the algorithm can easily be adapted for the reverse sizing. The issue is mostly notational. It seems as though Sadinle chose to label the datasets $X_1$ and $X_2$ such that $n_1 > n_2$ for mostly notational reasons. He uses the notation $Z_j = n_1 + j$ when record $j \in X_1$ has no match in $X_2$, which works because when $n_1 > n_2$, $n_1 + j$ never refers to any record in $X_2$. Jody uses the simplified notation $Z_j = n_1 + 1$ to refer to unmatched records, and this works for the same reason.
%However, when we relax the one-to-one constraint, or even encourage a modeller to allow multiple linkages in one dataset, we shouldn't denote the datasets according to simple notational convenience. This might require clever edits of the "Review of Prior Work" section, and other areas of the paper.
%Notationally, we can use  $Z_j = 0$ or $Z_j = -j$ as a way to unambiguously denote a record as nonmatching.
%Additionally, we could just strike the paragraph about the advantages of many-to-one matchings, and keep this strictly a paper about bipartite matching}



%\textcolor{blue}{This is a good point. I wonder if we could perhaps remove a lot of the notation by Sadinle/FS and redo the notation such that the ordering wouldn't matter. Also, it would make more sense to follow with the examples that we use in the paper -- NLTCS and EL Salvador so that we don't confuse the reader. Let's talk through this point as this would be a big change. If the algorithm and notation can be easily modified to reverse things, this would be a nice addition to the paper and improvement over Sadinle.}


\pointRaised{R1}{%
5. Section 3, full model: $I_obs(.)$ seems to be taken from a different paper dealing with missing fields, which is not at all discussed in this manuscript. This consideration of missing fields and the accompanying discussion of missing data mechanism should be included if such data is to be considered for this manuscript. This comment also applies to the $Gamma^{obs}$ notation in the following subsection.
}

\reply{%
	\textbf{Our missingness assumptions and the definition of $I_{obs}()$ have been added near end of Section 3.} 
}

\pointRaised{R1}{%
Section 3, full model: The Phi notation needs to be defined.
}

\reply{%
	\textbf{We have removed the $\Phi$ notation, using the parameters $\bm{m}$ and $\bm{u}$ throughout the paper.}
}

\pointRaised{R1}{%
	Section 3, full model: The square brackets seem to be misplaced, as the exponent includes indices of the sum contained in the brackets.
}

\reply{%
	\textbf{The expression is now corrected.}
}

\pointRaised{R1}{%
Section 3.1, I found the motivational descriptions in this section to be somewhat confusing, in that they seem to mix the posterior behavior (e.g., parameters are updated though standard multinomial-Dirichlet…) and model construction (as a function of pi, Z is a series of successes…). This subsection might be easier to understand if its purpose was more bluntly described in an opening sentence or paragraph and streamlined to focus only on the full conditional distributions for a Gibbs Sampler. In addition, the mixed motivation may have led to errors in intermediate conclusions. In particular, the derivation of the probability of $\Gamma_{.j}$ seems to be missing a factor of $u_{fl}^I(\gamma_{ij}^f=l)$. This is not important to the eventual full conditional of Z, but does not seem to be correct for the distribution of Gamma as claimed.
}

\reply{%
	We have revised our presentation of the Gibbs sampler with this feedback in mind. The main body of the text now presents the final full conditionals, with revised derivations now in the Appendix. In particular, we have included an extra step in the derivation of $p(\Gamma_{.j})$ that shows exactly how we divide by $\prod_{i = 1}^{n_1}\prod_{f = 1}^{F} \prod_{\ell = 1}^{L_f}  u_{fl}^{I(\gamma_{ij}^f=l)}$ and use proportionality to arrive that the distribution presented.
}

\pointRaised{R1}{%
Section 3.1, full joint posterior distribution: I was surprised to see that the contribution of the conditional distribution of Z was included using the summation-style notation rather than the binomial/multinomial-style multiplicative notation that naturally motivates the full conditional presented on the first line of page 8.
}

\reply{%
	\textbf{The contribution of $Z$ in the full conditional is just the prior distribution, which we have expressed with summation notation to reflect the piecewise nature of the prior. The binomial-looking expression at the top of page 8 is the posterior distribution for $\pi$. We note that the posterior $p(\pi| \Gamma, Z, \Phi)$ in fact only depends on $Z$, and is thus shortened to $p(\pi|Z)$. We have updated our explanation to make this more clear.}
}

\pointRaised{R1}{%
Section 3.1, first equation on page 8: There is an extra parenthesis in the exponent of (1-$\pi$).
}
\reply{%
	\textbf{This has been corrected.}
}

\pointRaised{R1}{%
Section 3.1, pmf for $Gamma_{.j}$ and full conditional for Z: The notation in the final equation may be more clear if you replace i with $z_j$, as in $w_{z_j j}$. I believe that “$n_1+1$” should be “$n_1+ j$” in several places.
}

\reply{%
	\textbf{We have changed $w_{ij}$ to $w_{z_j, j}$ in the pmf for $\Gamma_{.j}$ and the full conditional for $Z_j$. We have been sure that “$n_1+1$” has been changed to “$n_1+ j$” throughout the paper.}
}

\pointRaised{R1}{%
	Section 3.1, last equations on page 8: There seems to be some lack of specificity in describing these equations as full conditional distributions vs. steps in the Gibbs Sampler algorithm. It would be helpful to provide a brief justification of integrating out pi.
}
\reply{%
	\textbf{We have revised the presentation of the Gibbs sampler based on this feedback. The main body of the paper presents the results, and revised derivations (including justification for integrating out $\pi$) have been placed in the Appendix.}
}

\pointRaised{R1}{%
	Section 4.2, SEI procedure. I’m not sure I follow the description of the SEI procedure. (1) My understanding of SEI is that for patterns with lots of possible pairs, no single pair is likely to (should) be identified in the posterior distribution. (2) Thus, you take a small subsample of the records of size $S < H_{j_p}$, and store these in $R^{SEI}$ rather than the complete index of pairs in each R. (3) Can you quantify the computational savings due to this method? (4) Do you have advice for how small S can be? (5) In your simulations, how does the use of this method or choice of S affect the results?
}

\reply{%
	We have numbered each sentence in the comment above to clearly address each point. (1) and (2) are correct understandings of the SEI procedure. For (3), we quantify computational savings of SEI in Section 4.2, where we state:
	''Rather than storing \(n_1 \times n_2\) record labels, SEI allows us to store at most \(n_2 \times P \times S\) labels, regardless of how large \(n_1\) might be." SEI does not affect the computation time of the Gibbs sampler, since the complexity of the sampler is $O(n_2 P)$, regardless of how many comparison vectors are of pattern $P$. 
	
	For (4), we have included the following sentence in the paper ``In practice, we recommend $S=10$, as this reduces the number of stored indices for highly unlikely record pairings, but does is not likely to eliminate any of the indices for plausible matches."
	
	For (5), we have included the following sentence in the paper ``Choosing $S$ too low, like $S=1$ or $S=2$, can concentrate undue mass on unlikely matches and distort linkage results." 
}

\pointRaised{R1}{%
Section 4.4, Assumptions and definitions should be included in the statement of this lemma.
}

\reply{%
	\textbf{The lemma now includes assumptions and definitions for all variables used within the statement.}
}

\pointRaised{R1}{%
Section 5.1 simulations – there is a detailed comparison for one set of simulation settings, including a binary definition of a “match”. Do these results differ as these settings change?
}

\reply{%
	Regardless of the specific construction of the comparison vectors, the computational complexity the Gibbs sampler under BRL is $O(n_1 n_2)$, where as it is $O(n_2 P)$ for fabl. The exact shape of the lines will be different, but in all cases, fabl removes dependence on $n_1$ in the computational complexity. To illustrate this, we have included in the appendix an additional simulation study under different comparison vector settings.
}

\pointRaised{R1}{%
	Section 5.2 and 5.3 simulations – are these a single simulated datasets? Do these same results hold over multiple simulated datasets, Or was this just a chance result for each setting?
}

\reply{%
	In the caption under Figure 3, we explain that we have 100 pairs datasets for each level of overlap and error. Thus the table represents results from 900 pairs of datasets. We have added this description in the body of the paper as well for clarity. 
}

\pointRaised{R1}{%
Section 5.3 simulation – RR results are not presented.
}

\reply{%
	We have updated the Figure 4 with these results. Since it is presented alongside NPV and PPV, for which values close to 1 indicate strong performance, we have presented decision rate (DR) results, instead of rejection rate (RR) results.
}

\pointRaised{R1}{%
	Section 6.1 last sentences. I believe the last sentence should be removed, as it is redundant.
}
\reply{%
	\textbf{This has been removed.}
}
\pointRaised{R1}{%
	Section 6.2, second sentence. I believe the word “survey” is missing, as in “the initial SURVEY began…”
}
\reply{%
	\textbf{This has been fixed.}
}

\pointRaised{R1}{%
	Section 6.2. In the NLTCS, are participants added in later cohorts, or may we assume that every member of the smaller dataset should also be included in the earlier one?
}
\reply{%
	In each iteration of the study, some participants are added to the lists, and others are lost due to loss of contact or death. This has been clarified in the paper.
}

\pointRaised{R1}{%
	Figures 5 and 7 are not referenced in the text.
}
\reply{%
	\textbf{This has been fixed.}
}

	\clearpage
	\newpage

	\section*{Reviewer 3}
	\setcounter{responsectr}{0}

	The paper presents a computational variant of the method introduced by Sadinle (2017). Simulation results are presented and the variant is applied on a new, significantly larger data set. A significant part of the paper is a reproduction of the paper of Sadinle. The variant is introduced in section 4. It is simple but consequential. The simulations and applications show how the variant accelerates the computation relative to the original method and reveals the trade-offs.
	
	
	
	The value of the paper resides in that computational complexity continues to be a difficult obstacle in the application of Bayesian statistics. The proposals of the paper enable the application of Sadinle’s approach to an extent not possible before. It is important to present the research for what it is: An extension of the method of Sadinle motivated by computational considerations. Here Sadinle supersedes Fellegi-Sunter in the sense while the starting point of Sadinle’s research is Fellegi-sunter, the starting point of this paper is Sadinle’s work. This should be made clearer, beginning in the abstract and reiterated in the conclusion.

	\reply{%
	\textbf{Thank you for your positive feedback regarding our paper. We have clarified that we are motivation by the work of Sadinle and computational considerations. We have tried to make this more clear in both the abstract, introduction, and conclusion.}}
		

\newpage

\section{Additional Points Addressed}

We highlight some additional points addressed in our revision. \todo

\textcolor{blue}{Brian: These are just some suggestions that I have, so we can chat through these in case you don't agree! I have these marked up on a printed copy of the paper itself to help you more easily make the changes. Please note I have not made these changes, but I'm marking them on a marked up paper so you can go back through and make changes. It's likely that there may be more errors, so please go over things very carefully for all errors as the content is all there, we just have to weed out these tedious details.}

\textcolor{Red}{For the purpose of our revisions, I commented out the comments about minor typos. The comments that remain are things I want you to read so we can discuss.}

\begin{enumerate}
\item We have added introductory sentences to each section/subsection to provide more guidance to the reader. 
\begin{itemize}
	\item \textcolor{red}{I originally had sentences like this, and Jerry removed them. I also like it better without these introductory sentences, and don't find them necessary.}
\end{itemize}

\item The word databases is used throughout instead of files or lists. 
\begin{itemize}
	\item \textcolor{red}{Sadinle uses the words datafile and file. I'll remove database, and do that instead.}
\end{itemize}

%\item We have checked for the proper use of commas throughout the paper. 
%\begin{itemize}
%	\item \textcolor{red}{Done}
%\end{itemize}

%\item We have re-written our proposed work on page 2 to make it more clear our contributions to the literature as it was missing our improvements over existing work in terms of speed with minimal loss in accuracy as well as providing two case studies. One case study is from the original paper of our extension, while another case study illustrates our ability to scale to larger databases. We provide a more comprehensive discussion regarding our work. 
%\begin{itemize}
%	\item \textcolor{red}{I don't think that the proposed changes made the contribution more clear. I did other edits, adding the specifics }
%\end{itemize}

\item We have broken up sentences that are too long.
\begin{itemize}
	\item \textcolor{red}{We disagree on some of these. For all of the longer sentences that remain, I believe they are clearer together than broken up.}
\end{itemize}

%\item We have removed repetitive words, such as intuitive, additionally.  \textcolor{blue}{Brian: you can use something like antidote to check for these or go through yourself. Be careful with the word intuitive. While it might be obvious to you, it may not be obvious to the Editor and this word can be quite jarring). As an example, in my job talk, Mike West did not find the linkage structure to be intuitive and found it to be quite weird and confusing!}
%
%\item We have checked that we define terms before they are introduced.
%\begin{itemize}
%	\item \textcolor{red}{Done}
%\end{itemize} 

\item We have defined the parameters $m$ and $u$ on page 4. 
\begin{itemize}
	\item \textcolor{red}{I originally had the precise definition of $m$ and $u$ at this spot, giving the $m_{fl} = p(\gamma_{ij}^f = l | Z_{j} = i)$ definitions. However, technically, FS do not specify that comparison vectors be thresholded to discrete values, so this definition technically doesn't make sense here. I checked, and Sadinle defines the $m$ and $u$ parameters during the FS review section the same way that I do.}
	
	\textcolor{red}{I agree that reviewers might take note of this, so I could add the following sentence.}
	
	\textcolor{red}{"When using comparison vectors with discrete agreement levels as described above, the $\bm{m}$ parameter takes the form $\bm{m} = (\bm{m}_1, \ldots, \bm{m}_F)$, where $\bm{m}_f = (m_{f1}, \ldots, m_{fL_f})$ and $m_{fl} = P(\gamma_{ij}^f = l|Z_j = i)$ for all fields $f$ and agreement levels $l$. The $\bm{u}$ parameters are defined similarly, with $u_{fl} = P(\gamma_{ij}^f = l|Z_j \neq i)$."}
\end{itemize}

\item One page 5, section 2.2, we have made it more clear why our extension is warranted and needed in the literature. 
\begin{itemize}
	\item \textcolor{red}{Done.}
\end{itemize}

\item We have removed confusing language, such as "compactful representation of" on page 5.
\begin{itemize}
	\item \textcolor{red}{Done}
\end{itemize}

\item We have checked that fast beta linkage is represented by fabl throughout.
\begin{itemize}
	\item \textcolor{red}{Done. However, I think "fast beta prior" is fine, and I have left that as is.}
\end{itemize} 

\item \textcolor{blue}{See page 10 for a long list of comments.}
\begin{itemize}
	\item \textcolor{red}{I don't understand what this is.}
\end{itemize}

%\item We do not start sentences with a Greek letter or a parameter.
%\begin{itemize}
%	\item \textcolor{red}{Done.}
%\end{itemize}

%\item We have fixed the sentence ``sampling a value from...." that has a missing verb (see page 11).
%\begin{itemize}
%	\item \textcolor{red}{Done.}
%\end{itemize}
%\item We have fixed a typo in equation 6, which is a missing (). 
%\begin{itemize}
%	\item \textcolor{red}{Done. But this was a new typo introduced in my editing, not in the original submission. So I'll remove this one from the list.}
%\end{itemize}
%\item We have corrected many typos in section 5.1 page 12 (first paragraph). These include a comma that should not be present, a period instead of a comma, and a Greek letter that starts a sentence.
%\begin{itemize}
%	\item \textcolor{red}{Done.}
%\end{itemize}
%\item Break up the first long sentence in Lemma 1 as it's hard to read. 
%\begin{itemize}
%	\item \textcolor{red}{Done.}
%\end{itemize}
%\item "In the simulation results displayed..." This sentence is awkward and should be rephrased. Page 13. 
%\begin{itemize}
%	\item \textcolor{red}{Done.}
%\end{itemize}
\item Appendices should be done as A, B, C or the convention in BA (please check). 
\begin{itemize}
	\item \textcolor{red}{None of the articles I saw on the BA homepage had appendices. Some had "Supplementary Material," and those had numbered headings. So I'm leaving it as is for now. }
\end{itemize}

%\item There is a comma present that needs to be removed in Figure 2. 
%\begin{itemize}
%	\item \textcolor{red}{Done.}
%\end{itemize}
%
%\item Page 15, provide a reference to the Appendix.
%\begin{itemize}
%	\item \textcolor{red}{Already there.}
%\end{itemize}
%
%\item proportion should be proportional on page 15
%\begin{itemize}
%	\item \textcolor{red}{Fixed.}
%\end{itemize}
%\item datasets should be databases on page 17
%\begin{itemize}
%	\item \textcolor{red}{Done.}
%\end{itemize}
%\item bold m and u on page 18
%\begin{itemize}
%	\item \textcolor{red}{Done.}
%\end{itemize}
%\item Therefore needs a comma after it on page 18
%\begin{itemize}
%	\item \textcolor{red}{Done.}
%\end{itemize}
%\item equation 7 is not consistent with other equations on page 20
%\begin{itemize}
%	\item \textcolor{red}{Done.}
%\end{itemize}
%\item Convergence is stated on page 20. We have shown no issue of lack of convergence. Please address.
%\begin{itemize}
%	\item \textcolor{red}{Done.}
%\end{itemize}
\item Conclusion needs to be longer most likely or they will ask us to address this.
\begin{itemize}
	\item \textcolor{red}{I added a paragraph (in red), check it out!}
\end{itemize}
 
\item Reference summary of notation in main paper and make sure it is complete. 
\begin{itemize}
	\item \textcolor{red}{I included the summary of notation (one for the general Gibbs sampler, one for the hashing notation) in the main body.}
\end{itemize}

%\item Refer the reader back to the main body of appendix with a reference so it's easier to find the section. 
%\begin{itemize}
%	\item \textcolor{red}{Done.}
%\end{itemize}
%\item Provide an introduction to each subsection so it's more clear what you're presenting. Example: Section 7.2: In this section, we derive the full conditional distributions of section \ref{}. 
%\begin{itemize}
%	\item \textcolor{red}{See earlier response.}
%\end{itemize}
%\item Similar minor details in appendix that are marked up regarding punctuation/typos.
%\begin{itemize}
%	\item \textcolor{red}{Done.}
%\end{itemize}

\item \textcolor{red}{I moved the discussion of blocking and indexing the beginning of Section 4. It makes the most sense there, because that's when I introduce the computational speed ups.}


\item \textcolor{red}{Do I need to provide open source software? I don't really want to be responsible for having a perfectly functioning R package at this stage.}

\item \textcolor{red}{I got rid of both remarks. I figure if the information isn't relevant enough to be in the main flow of the paper, its probably not relevant enough to include.}

\item \textcolor{red}{I have for the moment removed the psuedo-code for fabl from the appendix. It didn't seem necessary, and the notation was difficult in that format.}

\end{enumerate}
\end{document}
