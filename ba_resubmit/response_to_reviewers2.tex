\documentclass[letterpaper, parskip]{scrartcl}

\usepackage{amsmath,amssymb,bm}
\usepackage{url}
\usepackage[hidelinks]{hyperref}
\usepackage[margin=1.4in]{geometry}
\usepackage{natbib}
\usepackage{xcolor}
\usepackage{tcolorbox}            % provides shaded box for reviewer comments
\usepackage{enumitem}             % use more compact lists
%\usepackage[parfill]{parskip}     % use skips between paragraphs and no indents
\usepackage{bibentry}             % provides \nobibliography command
\usepackage{xspace}

% Make itemize and enumerate environments more compact
\setlist[itemize]{noitemsep, topsep=0pt, parsep=0pt}
\setlist[enumerate]{noitemsep, topsep=0pt, parsep=0pt}

% -------- Define point raised command --------- %

\definecolor{darkpurple}{HTML}{500050}
\tcbuselibrary{breakable}       % allow tcolorbox to break across pages
\tcbuselibrary{skins}
\tcbset{%
	enhanced,%
	frame hidden,%
	sharp corners,%
	breakable,%
	parbox=false,%
	colback=lightgray!30,%
	enlarge left by=6pt,%
	width=\linewidth-6pt,%
	colback=gray!0,
	left=4pt,%
	right=0pt,%
	top=0pt,%
	bottom=0pt,%
	borderline west={2pt}{0pt}{lightgray},%
	%coltext=darkpurple,%
	before skip=14pt plus 2pt,%
	after skip=14pt plus 2pt}

%\newcommand{\pointRaised}[1]{%
%	\begin{tcolorbox}
%		\itshape #1
%	\end{tcolorbox}
%}
%
%\newcounter{responsectr}[section]     % counter for response resets at each section
%\newcommand{\reply}[2]{%
%	\refstepcounter{responsectr}%
%	\textbf{#1.\theresponsectr:} #2
%}

\newcommand{\pointRaised}[2]{%
	\textbf{#1.\theresponsectr:} #2
}

\newcounter{responsectr}[section]     % counter for response resets at each section
\newcommand{\reply}[1]{%
	\refstepcounter{responsectr}%
		\begin{tcolorbox}
			\itshape #1
		\end{tcolorbox}
}


\newcommand{\todo}{\textcolor{red}{[TODO]}\xspace}


\newcommand{\beka}[1]{{\color{teal}}}


\begin{document}

	% ****************** TITLE ****************************************

	\title{Response to Reviewer Comments}
	\maketitle
	

	
Thank you for the thoughtful feedback. We have made several broad changes to address the Editor's and Associate Editor's overall critiques, and have addressed each specific comment from the reviewer.

\section{AE Comments}

\pointRaised{AE}{I must add that reading the paper still feels like a chore, in part because the notation is so heavy, which may be hard to avoid, but also, in part, because some wording choices make certain sentences hard to parse, which can certainly be improved.}

%% RCS: Suggestion: Write a strong paragraph to the Editor after the paper is finalized. Come back to this. Explain that we have tried to simplify the notation when possible, however, in some places, the heavy notation is not possible. Thus, we have worked on improving more precise wording and removing sections or wording that is distracting or not needed. Thank the Editor and Associate Editor for helping us improve upon what we believe to be an important paper and breakthrough in the literature.

%% I would not make a large list of all the notation we have changed as it may cause the Editor to go picking over the paper more.

\reply{
\textcolor{teal}{We thank the Editor for the opportunity to revise our paper and for the suggestions. We have improved notation, when possible, however, as noted by the Editor in certain places, such as in the computational speeds ups of Section 4, this is difficult to avoid. We have done our utmost to simplify notation (when possible) and improve the parsing of sentences. Finally, we have attempted to clarify our overall contributions to the literature, which we highlight below. In addition, we provide responses to all the Editor's and Associate Editor's comments below.}


We propose fast beta linkage (\texttt{fabl}), which extends the \texttt{BRL} model for increased efficiency and scalability. We use independent priors for the matching status of each record, inducing a dependency that is stronger than the record pair independence in the original \cite{fellegi_theory_1969} but weaker that the one-to-one requirement of \cite{sadinle_bayesian_2017}. This independence assumption allows for a scalable update of the Gibbs sampler, contrasting sequential updates of \cite{sadinle_bayesian_2017}. We employ the decision theoretic  technique of \cite{sadinle_bayesian_2017} in order to ensure our method after record linkage is bipartite. This discovery allows us to (1) employ hashing techniques that speed up calculations and reduce computational costs, (2) compute the pairwise record comparisons over large data files via parallel computing, and (3) reduce memory costs via a novel method, storage efficient indexing. These contributions allow \texttt{fabl} to perform record linkage on much larger data files than previous Bayesian Fellegi-Sunter models at significantly increased speed with similar levels of accuracy. In particular, computation time under \texttt{BRL} grows quadratically, with the size of each data file, while computation time under \texttt{fabl} grows linearly, only with the size of the smaller data file. Finally, we provide two real illustrations of our methodology and simulation studies with reproducible software.

}

%\reply{We have made several efforts to simplify notation in this revision.
%
%\begin{itemize}
%	\item Instead of referring to files as $\mathbf{X}_1$ and $\mathbf{X}_2$, we are now using $A$ and $B$. This particularly simplifies the notation in Section 4.2.
%	
%	\item At the end of Section 4.1, we have rewritten the likelihood function in terms of the agreement patterns and the summary statistics.
%	This likelihood is then used throughout Section 4.2 in the discussion of efficient posterior inference. We hope this expression clarifies the meaning of "preservation of weight (Comment 5), and strengthens the argument for $\tilde{\Gamma} = \{\mathcal{P}, \mathcal{R}, \mathcal{N} \}$ being summary statistics for $\bm{m}$, $\bm{u}$, and $\bm{Z}$ (Comment 4). 
%	
%	\item To denote when the $(i,j)$ record pair exhibits agreement pattern $p$, we have removed the $(i, j) \in h_p$ notation for $\gamma_{ij} = h_p$.
%	
%	\item In Section 4.1, we have provided a specific example of a comparison vector and how the hashing procedure works. 
%	
%	\item Section 4.1 introduces the summary statistic $N_{j_p}$. After careful review, we realized that by changing the order of subscripts and instead using $N_{p_j}$, we can replace the sum of counts across all records (previously denoted $H_p$), with the more intuitive $N_p$. Thus, we have removed all of the ''$H$" quantities throughout Section 4. 
%\end{itemize}
%}

\pointRaised{AE}{Line 80 on p. 3 says: “For ease of readability, we follow the convention established by \cite{sadinle_bayesian_2017} and say “record $i \in X_1$” rather than the more compact $x_{1i}$.” Taking this at face value implies that, in any sentence, one could swap “record $i \in X_1$” for “$x_{1i}$,” which clearly is not true.  So, as it stands, the sentence does little to improve ease of readability.  In fact, if the point is that “$x_{1i}$” will never be used again, why is the notation introduced in the first place?}

%%% RCS: Thank the Editor, explain the change. I have removed repetition and shortened this bit. 

\reply{
We thank the Editor for bringing up this point. In order to simplify notation, we refer to the two data files as
 $A$ and $B$, where the updated material reads:
	
	``Consider a data files $A$ and $B$, consisting of records $A_i$ and $B_j$ respectively, where $i \in \{1, \ldots n_A\}$ and $j \in \{1, \ldots n_B\}$". See Section 2, page 3. 
}

\setcounter{responsectr}{0}
\section{Reviewer Comments}

	\pointRaised{R}{%
	For the missing data treatment now included just before Section 2.1, is the assumption truly missing at random, or missing completely at random? Or, does this distinction not matter because of the independence assumed across elements of the comparison vector?}
	
	
	\reply{\textcolor{teal}{The distinction does not matter due to assuming that the comparison vectors are conditionally independent given the coreference matrix (or matching label). We now state the following in Section 3, page 5:}
	
	\textcolor{teal}{``Assume the comparison vectors are conditionally independent given the coreference matrix (or matching label) and that missing comparisons are  missing at random (MAR) or missing completely at random (MCAR). Due the the conditional independence assumption, we can marginalize over the missing data in (6a) and do all computation simply using the observed data. For details, refer to \cite[Section 6.2]{LittleRubin2002}, \cite[Section 3.1]{sadinle_detecting_2014}, and \citep[Section 4.2]{sadinle_bayesian_2017}.}
	
%	Furthermore, both MAR and MCAR lead to assuming the missing data is ignorable in this situation, which implies that the marginalized likelihood under all three assumptions is the same."
	
%	``Assume the comparison vectors are conditionally independent given the coreference matrix (or matching label) and that missing comparisons are ignorable or missing at random (MAR, per \cite{LittleRubin2002}). With either the ignorability or MAR assumption, we can marginalize over the missing data in (6a) and do all computation simply using the observed data. For details, refer to \cite[Section 6.2]{LittleRubin2002} and \cite[Section 3.1]{sadinle_detecting_2014}, and \citep[Section 4.2]{sadinle_bayesian_2017}. Due to the conditional independence assumption, the missingness mechanism is equivalent to being missing completely at random (MCAR, per \cite[p. 12]{LittleRubin2002}). To summarize, we can assume the missingness is ignorable, MAR, or MCAR in this situation."
%	
	
	
%	(MAR, per \cite{LittleRubin2002}). With either the ignorability or MAR assumption, we can marginalize over the missing data in (6a) and do all computation simply using the observed data. For details, refer to \cite[Section 6.2]{LittleRubin2002} and \cite[Section 3.1]{sadinle_detecting_2014}, and \citep[Section 4.2]{sadinle_bayesian_2017}. Due to the conditional independence assumption, the missingness mechanism is equivalent to being missing completely at random (MCAR, per \cite[p. 12]{LittleRubin2002}). To summarize, we can assume the missingness is ignorable, MAR, or MCAR in this situation."
	}
	
	%% RCS: See two comments below regarding MAR versus MCAR. I have an alternative response for the one regarding miss at random. 
	
%	\textcolor{teal}{Answer for MAR: Yes, the assumption is truly missing at random. This distinction between truly missing at random and missing completely at random does not matter due to two assumptions. Following Sadinle (2017), we assume that the fields in the comparison vectors are conditionally independent given the coreference matrix and we assume that the missing comparisons are ignorable (Sadinle (2017). Therefore, our likelihood is the same as in Sadinle (2017) (see equations (3) and (4) of his paper). As stated by Sadinle (2017), ``we work under the assumption of ignorability of the missingness mechanism for the comparisons so that we can base our inferences on the marginal distribution of the observed comparisons (Little and Rubin, p. 90)."  We refer to Sadinle (2017), Section 4.2, page 604 for further details.}
	
%	\reply{The assumption is missing completely at random (MCAR) due to fact that we assume the fields in the comparison vectors are conditionally independent given the coreference matrix (or matching label). The assumption is not missing at random (MAR) as we do not assume that the missing comparisons are also ignorable. See Section 3, page 5 regarding clarification of this point \citep{LittleRubin2002}.}
%	\textcolor{red}{Still checking on this and looking at Little and Rubin.}
	
	

%	\reply{% 
%		The assumption is missing completely at random. You are correct that this is a result of the independence assumed across elements of the comparison vector. This has been clarified. 
%	}

	\pointRaised{R}{%
	Appendix 8.2: I appreciate the streamlining of the discussion of the proposed algorithms and think it was a good choice to move the full derivation to an appendix. However, I do not follow the re-expression of the pmf for $\Gamma_{.j}$. First, the opening square brackets are still misplaced
	throughout. In the second line of this derivation, the authors divide the expression by a product of the element-wise conditional probabilities of a match in the comparison vector (u), raised by an indicator that the element of the comparison vector equals a particular value. Perhaps I misunderstand, but I believe this product is not constant in $\Gamma_{.j}$ or $u$, and so the total expression is not proportional to the line above, as interpreted as a pmf. I think that the end effect is that the last line of the newly expressed pmf is missing a factor of $\prod_i \prod_f \prod_l u_{fl}^{I(\gamma_{ij}^f = l) I_{obs}(\gamma_{ij}^f)}$ regardless of the value of $z_j$.}
	
%% Don't apologize. Just answer the questions very completely. 	
	
%\textcolor{teal}{Don't apologize!}

\reply{%
	The square brackets have been corrected. We have clarified the derivation of the full conditional distribution of $Z_j$, separately handling the case when $B_j$ has a match and when it does not. Refer to Appendix B, page 26 -- 27 regarding this extended derivation.
	}


%\reply{%
%	The square brackets have been corrected, and we apologize for the oversight. We have clarified derivation of the likelihood for $Z_j$, separately handling the case when $B_j$ has a match and when it does not.
%}

	\pointRaised{R}{%
Appendix 8.2: Thank you for including the details on integrating out pi from the full conditionals. They surprised me. I had assumed that the authors had integrated out pi in the prior for Z, as this is what Sadinle (2017) had done to form the “beta prior for bipartite matchings”. (1) Does the alternate approach presented in this paper provide a different algorithm than directly integrating the prior distribution? (2) Is the presented alternative approach justified?}



\reply{No, we do not integrate out $\pi$ as done in \cite{sadinle_bayesian_2017} as this would result in a sequential sampler as opposed to the parallel one that we propose. This is one of our contributions to our paper. 

(1) Yes, our alternative approach is different than directly integrating over the prior distribution $\pi.$ We have attempted to make this more clear in our revised version. In short, our prior distribution over $Z$ leads to a Gibbs sampler that leads to parallel updates, which contrasts that of Sadinle (2018). Of course, we could integrate out $\pi$ in our approach, however, this would lead to sequential updates, which would be slow in practice. See Appendix B, page 27. 

(2) Yes, our alternative approach is justified. We provide a justification in Appendix B, page 27 regarding why the updates are independent and not sequential. Thank you for the excellent questions, which have greatly improved our paper. 

}



%
%\reply{%
%	We realize that the integration of pi out of the full conditionals is not necessary for our sampler. For clarity, we decided to redo the simulations using a Gibbs sampler that samples $\pi$ from its full conditional, and then samples $Z$ from its full conditionals. We obtained equivalent results, which is unsurprising, because with high numbers of observations, the posterior of the beta distribution is highly concentrated around its mean.
%	
%	Therefore, we have updated the paper with full conditionals for the standard Gibbs sampler. We leave the remark about marginalization in the Appendix to facilitate comparisons to Sadinle's method.
%
%}


\pointRaised{R}{%
	I do not understand the statement just below Equation (10): “When j has no match in $X_1$, we write $(n_1 +j, j) \in h_{P+1}$” My understanding of these patterns is that they are based on observed comparison vectors without consideration of Z (matches). In the second paragraph of Section 4.3, the H notation includes the matches (Z), in notation and definition that seems to conflict with the statement just below Equation (10). This also comes into play in Equation (16)}
	
	
	
	%% RCS: Tried to answer this from the original paper and the updated paper. Tried to make it shorter and close it out very quickly. 
	

	
	\reply{Yes, in the original manuscript, the comparison vectors are created and therefore, patterns are assigned) without regard to $Z$. Given the confusion caused to the Associate Editor, we removed the sentence.	}

%	\reply{%
%	You are correct that the comparison vectors are created (and therefore, patterns are assigned) without regard to $Z$. This notation was created purely to be able to denote when a record was left unlinked during the Gibbs Sampler as in Equation 16 in the submitted draft. 
%	
%	I can see why this is confusing. Therefore, we have removed that line, and changed the notation for record $B_j$ being unmatched to just be''otherwise".
%	}

\pointRaised{R}{%
	Third paragraph of Section 4.1: The authors claim they are computing “sufficient statistics”. What exactly are these statistics sufficient for?
	}
	
	%%Comment: Make it more brief and no need to provide notation when it's not needed.
	
	\reply{
	Thank for pointing out our typo. We have revised the text to state ``summary statistics" instead of ``sufficient statistics."
	}
	
	

%	\reply{%
%		We show in the revised Section 4.2 that we can write conditional likelihoods  and posterior updates for $\bm{m}$, $\bm{u}$, and $\bm{Z}$ using the statistics in $\tilde{\Gamma} = \{\mathcal{P}, \mathcal{R}, \mathcal{N} \}$. However, we cannot express the full likelihood in (14) through these statistics. Therefore, we have revised to text to say ``summary statistics" instead of  ``sufficient statistics."}

\pointRaised{R}{%
	First paragraph of Section 4.2: The authors state: “Posterior calculations still attribute the appropriate weight to all records through the summary statistics…” What is meant by the term “weight”? Which records are appropriately weighted – those in X2?}
	
		%% Comment: Brian, you have not answered the reviewer's question regarding what is meant by weight and what records are appropriately weighted. \textcolor{red}{Please revise to directly answer the question in the revised text with equation numbers. To my understanding, the weights have been removed and section 4 has been rewritten }}
	\reply{
	\textcolor{teal}{In the original manuscript, we consider all records in the first file ($X_{1}$) that share an agreement pattern with the second data file ($X_{2}$). In this situation, all these records have the same Fellegi-Sunter likelihood ratio weight, $w_{ij}$. Given this result, we denoted the weight $w_p$ since these records map to the same agreement pattern.} 
	
	\textcolor{teal}{In the revised manuscript, we define $m_p$ and $u_p,$ which are the 
	 probabilities that records $A_i$ and $B_j$ form agreement pattern $p$ given that
 they are a match and non-match, respectively. For each pattern $p,$ we define $w_p = m_p/u_p.$ We hope that this will be more clear than our previous presentation of the material. See Section 4.1, pages 8--9.}
 
 
 }
 

	

	
	

	
%	\textcolor{teal}{
%	We have revised the ordering of Sections 4.2 and 4.3, so that we first discuss hashing, then posterior inference, then chunkwise computation of the comparison matrix, and then SEI. With this new ordering, and with the newly provided equations 15 -19, we hope that the contribution of \emph{all} record pairs is recorded through the summary statistics in $\mathcal{N}$ is now more clear. 	
%	}

%\reply{%
%	
%	We have switched the order of Sections 4.2 and 4.3, so that we first discuss hashing, then posterior inference, then chunkwise computation of the comparison matrix, and then SEI. With this new ordering, and with the newly provided equations (15 -19) it is clearer that the contribution of \emph{all} record pairs is recorded through the summary statistics in $\mathcal{N}$.
%}

\pointRaised{R}{%
	Second paragraph of Section 4.2: “and delete those comparison vectors”. Which are “those” vectors?}
	
	\reply{
	
%	In the revised paper, ``vector" is replaced with ``matrix" when referring to $\gamma.$ (See R.13 regarding making the paper consistent below).
	
	\textcolor{teal}{In the original manuscript, ``those comparison vectors" refer to removing the larger (and expensive) $\Gamma^{ab}$ from memory and continuing our calculations with the compressed comparison vectors $\tilde{\Gamma}^{ab}.$ }
	
	\textcolor{teal}{The revised text states: ``Then, we conduct hashing, obtain the compressed comparison matrix, $\tilde{\gamma}^{ab}$, and remove the memory-intensive comparison matrix, $\gamma^{ab}$, before continuing with the next chunk of data. See Section 4.3, page 10. }
	
%	 \textcolor{red}{Question for Brian: is lowercase $\gamma$ always a matrix? Let's make sure we're not swapping between vectors and matrices. In the previous version this was a vector and now its listed as a matrix. I would argue that a matrix should use capital notation, such as $\Gamma.$}
	}

%\reply{%
%	We have revised that sentence to read ''We then conduct hashing, obtain the compressed $\tilde{\Gamma}^{ab}$ for later calculations, and delete the larger $\Gamma^{ab}$ from memory before continuing with the next chunk of data."
%}



\pointRaised{R}{%
	Where does $R^{SEI, cd}$ come into play in the partitioned algorithm presented in Equations (13) and (14)? I recommend that the authors either refrain from suppressing the SEI notation or further explain how the SEI algorithm has changed the quantities in these equations.}
	
\reply{
	\textcolor{teal}{In order to attempt to make the SEI algorithm more clear, we have revised the ordering of Sections 4.2 and 4.3, so that 1) hashing, 2) posterior inference, 3) chunkwise computation of the comparison matrix, and 4) SEI. With this new ordering, and with the newly provided equations 15 -19, we hope that the contribution of \emph{all} record pairs is recorded through the summary statistics in $\mathcal{N}$ is now more clear. In addition, we hope that the SEI algorithm updates will now be more clear that the updates in (16a), (16b), and (18) only depend on $\mathcal{N}$. The SEI algorithm affects the step shown in (19). 
	}
	}	
	
	


%\reply{%
%	We have reordered Section 4 so that all posterior inference is presented before SEI. This makes it clear that the posterior updates in (16a), (16b), and (18) depend only on $\mathcal{N}$. SEI only affects the step shown in (19).
%}

\pointRaised{R}{%
	Section 4.2: I appreciate the practical advice about choosing S for the SEI method. However, this choice seems arbitrary in the absence of further discussion/evidence. Given that the primary novelty of the manuscript is in methods to speed and otherwise improve computation, I am surprised that this aspect of computational innovation is presented with virtually no theoretical or empirical exploration. Presumably the SEI method has some sort of accuracy trade-off, as the authors warn that linkage results may be “distorted” if S is low. However, this trade-off is not quantified or even discussed in practical terms beyond the terse recommendation to choose S=10.
}


\reply{% 
	Refer to Section 5.3, which explores the trade-offs regarding different choices of $S$ and discusses this in practical terms. 
}
\pointRaised{R}{%
	Page 5, 2 sentences before equation (4): I believe the sum should be of $I(Z_j \leq n_1)$, not $I(Z_j \leq n_1 +1)$.}

\reply{%
	This has been corrected; thank you.
}

\pointRaised{R}{%
	Equation (6a): The indices do not match the subscripts in the indicator function in each summand, or their standard meaning in table 1.}

\reply{%
	\textcolor{teal}{The typo has been fixed.}
}

\pointRaised{R}{%
	Generally, the authors seem to arbitrarily use upper and lower case z interchangeably in function definitions.}

\reply{%
	We use $Z$ when discussing a random quantity, and $z$ to reference a realized value. We have revised all instances of inconsistency. 
}

\pointRaised{R}{%
Equations (8) and (9): Should the weights have superscript (s) (as the Zs do)?}

\reply{%
	This has been fixed. 
}

\pointRaised{R}{%
	Gamma is in some places described as a set and in others as a matrix (particularly in sections 4.2 and 5.1).
}

\reply{%
	\textcolor{teal}{Note that $\gamma$ is a comparison matrix comprised by comparison vectors. All references to $\gamma$ as a set have been removed. See Section 2, page 3, where we define both the comparison vector and comparison matrix.}
}

\pointRaised{R}{% 
	Section 4.3, second paragraph, the definitions of the concatenated vectors $\alpha_0$ and $\beta_0$ should have final elements subscripted by $L_f$, where the sub-subscript is capitalized.}

\reply{%
	This has been revised; thank you.
}


	\clearpage
	
	\bigskip
	
	\bibliographystyle{jasa}
	\bibliography{biblio}

\end{document}
