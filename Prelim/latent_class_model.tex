\documentclass{beamer}
\usetheme{Rochester}
\addtobeamertemplate{navigation symbols}{}{ \hspace{1em}    \usebeamerfont{footline}%
	\insertframenumber / \inserttotalframenumber }
\begin{document}
	\title{Latent Class Models}
	%\subtitle{Using Beamer}
	\author{Brian Kundinger}
	\date{\today}
	
%	\AtBeginSection[]{
%		\begin{frame}
%			\vfill
%			\centering
%			\begin{beamercolorbox}[sep=8pt,center,shadow=true,rounded=true]{title}
%				\usebeamerfont{title}\insertsectionhead\par%
%			\end{beamercolorbox}
%			\vfill
%		\end{frame}
%	}


%	\begin{frame}
%	\frametitle{Outline}
%	\tableofcontents
%	\end{frame}

\section{Latent Class Models}

	\begin{frame}{Latent Class Models}
	\begin{itemize}
		\item In Dunson's STA 841 class "Categorical Data Analysis," we learned about a \emph{latent class models}
		\item In class, we used to them to classify DNA sequences, but I think they may be effective in classifying individuals based on their travel path in a network
		\item DNA sequences have complicated dependencies. The idea here is to model the sequences such that by conditioning on latent classes, the expressions in each position are independent.  
		
		
		\item DISCLAIMER: I see many shortcomings of the model, but just wanted to share it as a potential idea
	\end{itemize}
	\end{frame}

\begin{frame}{Data}
	\begin{itemize}
		\item Individuals $i \in \{1, \ldots, I \}$
		\item Nucleotide position $p \in \{1, \ldots, P \}$
		\item Nucleotide expression  $y_{ip} = d \in \{A, C, G, T\}$
		\item Expert classification $e_i \in \{IE, EI, N\}$
		
		<2->\item We attempt to recover the expert classification through latent class $z_i = k \in \{1, \ldots, K\}$. Number of classes $K$ can be fixed or estimated nonparametrically
		<2->\item Vector of expression probabilities $\theta_{kp} = (\theta_{kpA}, \ldots, \theta_{kpT})$ for each class and position
	\end{itemize}
\end{frame}

\begin{frame}{Data}
	\begin{itemize}
		\includegraphics[width = \textwidth, height = .7\textwidth ]{data_snapshot.png}
	\end{itemize}
\end{frame}


\begin{frame}{Model Formulation}
	\begin{itemize}
		\item $y_{ip} \sim \text{Multinomial} (\theta_{z_i, p})$
		\item $z_i \sim \text{Dirichlet}(\alpha_1, \ldots, \alpha_K)$
		\item $\theta_{z_i, p} \sim \text{Dirichlet}(\beta_1, \ldots, \beta_D)$
		<2->\item $z_i$ can also be modeled through a Dirichlet process stick-breaking prior so that the number of clusters is inferred through the data
		<3-> Parameters are estimated through straightforward Gibbs sampler
	\end{itemize}
\end{frame}

\begin{frame}{Result}
	\begin{itemize}
		\includegraphics[width = \textwidth, height = .7\textwidth ]{latent_variable_plot.png}
	\end{itemize}
\end{frame}


\end{document}